\addcontentsline{toc}{chapter}{Abstract}
\begin{center}
\textbf{\large Abstract}
\end{center}

This thesis summarize and helps to understand, which advantages and
disadvantages provides static type checking in contrast to dynamically typed
languages. Most of figured out benefits are implemented using clojure.spec
annotations and other related tools. Drawbacks of static type checking are
reduced to minimum.

This benefits are important and not only Clojure developers care about them.
Python type hints, TypedRacket, TypedClojure, TypeScripts, many other languages,
gradual typing frameworks and related researchs shows that problems exist and
deserve attention.

Clojure has a pretty smart community and unusual design decision behind language
implementation. Due to rich ecosystems of hosted platforms (jvm, clr, js), very
vibrant community, simplicity and extensibility of the language it is much
easier to implement ideas for solving such problems. Code as data idioms
significantly helps in this.

Most of benifits borrowed from static world is achivied for pretty sane cost.
Static analysis tools are not mature and not natural for dynamic languages. Now,
it is hard to say how this technologies and techniques will be adopted in
industry, but they looks promising.
