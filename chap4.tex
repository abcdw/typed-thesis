\chapter{Evaluation}

\label{chap:evaluation}

% \section{General}
In previous chapter was introduced examples, which shows how to enrich
documentation in IDE, how to sample data, generate tests and even do a static
analysis, but nothing is giving for nothing, everything has a cost. There was a
strict constraint about keeping old workflow and seamless integration of new
tools. From developers point of view changed not much. Most noticeable change
that it is necessary to run static analysis tools, but it optional and can be
done via version control system or continuous integration service and another
important change that specs should be written for existing and future code.
Other aspects as IDE performance or additional steps in workflow did not
significantly affected by new tools.

To understand how much additional code developers will write for utilizing power
of optional annotations using clojure.spec Clojure core was analyzed. Using
clojure reader specification \cite{readerspec} was created grammar for JavaCC.
Source code provided in the Chapter \ref{chap:appendix}.

% \section{Amount of code}

Java Compiler Compiler (JavaCC) is the most popular parser generator for use
with Java applications. A parser generator is a tool that reads a grammar
specification and converts it to a Java program that can recognize matches to
the grammar. In addition to the parser generator itself, JavaCC provides other
standard capabilities related to parser generation such as tree building (via a
tool called JJTree included with JavaCC), actions, debugging, etc.

This grammar with auxiliary functions inside it allows to generate parser, which
can gather some metrics from source code. Using this parser two version of
clojure core library (with and without specs) was compared. Clojure has no
objects, classes, inheritance or variables and consist of S-expressions that is
why it is impossible to compare code using standard metrics, therefore following
metrics can be unusual for people not familiar with Lisp. Results and
explanation provided below.



Lines of code is not the best metric to represent size of code in Clojure,
that is why another metric was used for those needs. It is a number of form. In
Clojure code is a bunch of nested data structures, which contains some terminal
tokens. Such terminals called symbols. Form is a symbol or data structure: list,
map, vector, set. Other unusual metric is a maximum form nestiness, which
represents deepest data structure in the code. Other metrics should be clear
from there names.

Columns are simple, first for metrics values of code without specs, second for
values of code with specs and the last one for difference between annotated code
and code without annotations calculated using formula: $(with - without)/with$.
Statistics should be representative as it gathered from codebase of Clojure
language core library.

As definitions sections presents: code for spec does not add any function to the
code, it only adds one macro probably to simplify some routine, one macro is not
significant for our case. This result is predictable and just provide evidence
that logic and structure of existing code does not changed.

On the other hand amount of keywords are noticeably varies between two versions
of the code. $17.6\%$ is a pretty big addition, but it is also expected.
Clojure.spec strongly relies on keywords for registering specs. Overall codebase
without annotations will be smaller only by $3.3\%$. It seems like a sane trade
off for all features provided by this tools.


\begin{table}[]
\centering
\caption{Code decrease}
\label{}
\begin{tabular}{lrrr}
                                                   & \multicolumn{1}{l}{}         & \multicolumn{1}{l}{}       & \multicolumn{1}{l}{}          \\ \cline{2-4}
\multicolumn{1}{l|}{}                              & \multicolumn{1}{l|}{without} & \multicolumn{1}{l|}{with}  & \multicolumn{1}{l|}{decrease} \\ \cline{2-4}


                                                   & \multicolumn{1}{l}{}         & \multicolumn{1}{l}{}       & \multicolumn{1}{l}{}          \\
\multicolumn{4}{c}{Forms}                                                                                                                      \\ \hline
\multicolumn{1}{|l|}{forms}                        & \multicolumn{1}{r|}{24266}   & \multicolumn{1}{r|}{25087} & \multicolumn{1}{r|}{0.0327}   \\ \hline
\multicolumn{1}{|l|}{symbols}                      & \multicolumn{1}{r|}{14772}   & \multicolumn{1}{r|}{15068} & \multicolumn{1}{r|}{0.0196}   \\ \hline
\multicolumn{1}{|l|}{literals}                     & \multicolumn{1}{r|}{3388}    & \multicolumn{1}{r|}{3692}  & \multicolumn{1}{r|}{0.0823}   \\ \hline
\multicolumn{1}{|l|}{keywords}                     & \multicolumn{1}{r|}{1405}    & \multicolumn{1}{r|}{1705}  & \multicolumn{1}{r|}{0.1760}    \\ \hline
\multicolumn{1}{|l|}{maximum form nestiness}       & \multicolumn{1}{r|}{17}      & \multicolumn{1}{r|}{17}    & \multicolumn{1}{r|}{0}        \\ \hline

                                                   & \multicolumn{1}{l}{}         & \multicolumn{1}{l}{}       & \multicolumn{1}{l}{}          \\
\multicolumn{4}{c}{Collections}                                                                                                                \\ \hline
\multicolumn{1}{|l|}{lists}                        & \multicolumn{1}{r|}{6330}    & \multicolumn{1}{r|}{6525}  & \multicolumn{1}{r|}{0.0299}   \\ \hline
\multicolumn{1}{|l|}{vectors}                      & \multicolumn{1}{r|}{1741}    & \multicolumn{1}{r|}{1748}  & \multicolumn{1}{r|}{0.0040}    \\ \hline
\multicolumn{1}{|l|}{maps}                         & \multicolumn{1}{r|}{646}     & \multicolumn{1}{r|}{648}   & \multicolumn{1}{r|}{0.0030}    \\ \hline
\multicolumn{1}{|l|}{sets}                         & \multicolumn{1}{r|}{39}      & \multicolumn{1}{r|}{58}    & \multicolumn{1}{r|}{0.3276}   \\ \hline


                                                   & \multicolumn{1}{l}{}         & \multicolumn{1}{l}{}       & \multicolumn{1}{l}{}          \\
\multicolumn{4}{c}{Definitions}                                                                                                                \\ \hline
\multicolumn{1}{|l|}{public function defenitions}  & \multicolumn{1}{r|}{475}     & \multicolumn{1}{r|}{475}   & \multicolumn{1}{r|}{0}        \\ \hline
\multicolumn{1}{|l|}{private function defenitions} & \multicolumn{1}{r|}{30}      & \multicolumn{1}{r|}{30}    & \multicolumn{1}{r|}{0}        \\ \hline
\multicolumn{1}{|l|}{macro definitions}            & \multicolumn{1}{r|}{72}      & \multicolumn{1}{r|}{73}    & \multicolumn{1}{r|}{0.0137}   \\ \hline
\multicolumn{1}{|l|}{multimethods defs}            & \multicolumn{1}{r|}{3}       & \multicolumn{1}{r|}{3}     & \multicolumn{1}{r|}{0}        \\ \hline
\multicolumn{1}{|l|}{multimethods implementations} & \multicolumn{1}{r|}{2}       & \multicolumn{1}{r|}{2}     & \multicolumn{1}{r|}{0}        \\ \hline
\end{tabular}
\end{table}