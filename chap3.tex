\chapter{Implementation}
\label{chap:implementation}

\section{What we want to achive}

Better Communication. Clojure is a dynamic language, and thus far we have relied
on documentation or external libraries to explain the use and behavior of
functions and libraries. But documentation is difficult to produce, is
frequently not maintained, cannot be automatically checked and varies greatly in
quality. Specs are expressive and precise. Including spec in Clojure creates a
lingua franca with which we can state how our programs work and how to use them.

More Leverage and Power. A key advantage of specifications over documentation is
the leverage they provide. In particular, specs can be utilized by programs in
ways that docs cannot. Defining specs takes effort, and spec aims to maximize
the return you get from making that effort. spec gives you tools for leveraging
specs in documentation, validation, error reporting, destructuring,
instrumentation, test-data generation and generative testing.

Improved Developer Experience. Error messages from macros are a perennial
challenge for new (and experienced) users of Clojure. Specs can be used to
conform data in macros instead of using a custom parser. And Clojure’s macro
expansion will automatically use specs, when present, to explain errors to
users. This should result in a greatly improved experience for users when errors
occur.

More Robust Software. Clojure has always been about simplifying the development
of robust software. In all languages, dynamic or not, tests are essential to
quality - too many critical properties are not captured by common type systems.
spec has been designed from the ground up to directly support generative testing
via test.check. When you use spec you get generative tests for free.

Taken together, the features of spec demonstrate the ongoing advantages of a
powerful dynamic language like Clojure for building robust software - superior
expressivity, instrumentation-enhanced REPL-driven development, sophisticated
testing and more flexible systems.

\section{Usage of clojure.spec}
Explain in-dept capabilities of clojure.spec and how to use them for solving
problems described above
\subsection{sub1}
\subsection{sub2}

\section{Run-time docstring enrichment}
Description of implementation of run-time docstring enrichment using function
meta-data

\section{Automated test generation}
Explanation of implementation of macros, which generates tests based on existing
code and optional annotations.
