\documentclass[a4paper,14pt]{extreport}
\usepackage[utf8]{inputenc}
\usepackage[english,russian]{babel}
% \usepackage[dvips]{graphicx}
\usepackage[pdftex]{graphicx}
\usepackage{geometry}
\geometry{left=25mm}
\geometry{right=20mm}
\geometry{top=20mm}
\geometry{bottom=20mm}
\graphicspath{{image/}}
\usepackage[usenames]{color}
\usepackage{colortbl}
\usepackage{multirow}
\usepackage{amssymb,amsfonts,amsmath,mathtext,cite,enumerate,float}
\usepackage{setspace}
\usepackage{indentfirst}
\onehalfspacing


\begin{document}
\begin{titlepage}
\begin{table}[]
    \centering
    \begin{tabular}{rcl}
    Автономная некоммерческая &
    \multirow{4}{*}{\includegraphics[width=40mm]{image/logo.eps}}
          & Autonomous noncommercial \\
    организация высшего  & & organization of higher \\
    образования & & education \\
    «Университет Иннополис»  &
     & «Innopolis University» \\
    \hline
    \hline
    \end{tabular}
    \label{tab:my_label}
\end{table}
\vline
\vspace{20mm}

\begin{center}
\textbf{АННОТАЦИЯ \\ НА ВЫПУСКНУЮ КВАЛИФИКАЦИОННУЮ РАБОТУ  \\
ПО НАПРАВЛЕНИЮ ПОДГОТОВКИ \\ 09.03.01 --- «ИНФОРМАТИКА И ВЫЧИСЛИТЕЛЬНАЯ ТЕХНИКА»}
\end{center}
\vspace{20mm}


    \begin{tabular}{ll
|>{\columncolor[gray]{.9}}l|}
\cline{3-3}
\textbf{Тема:} &
     &
    \makebox[133mm][l]{Альтернативы статической типизации для Clojure}    \\
    &&\\
    && \\
    &&  \\
\cline{3-3}
    \end{tabular}
\vspace{5mm}


    \begin{tabular}{ll
|>{\columncolor[gray]{.9}}l|l
|>{\columncolor[gray]{.9}}l|}
\cline{3-3} \cline{5-5}
Выполнил &
     &
    \makebox[77mm][l]{Тропин Андрей Геннадьевич}   &
    &    \\
    &&&&
    \makebox[39.5mm]{\textcolor[gray]{.7}{подпись}} \\
    &&&& \\
    \cline{3-3} \cline{5-5}
    \end{tabular}
\vspace{5mm}

\vspace{\fill}

\begin{center}
Иннополис, 2017
\end{center}
\end{titlepage}


\newpage
% \noindent {\large \textbf{Содержание}} \\

\chapter*{Содержание}
\documentclass[dvips,letterpaper,12pt]{report}
\usepackage{thesis}

\begin{document}

\pagenumbering{roman}

% Fill in the title, author, degree name, department, and month/year.
% Upon completion, this should look like the following:
%\thesistitle
%	{Complicated and Important-Sounding Thesis Title}
%	{John P. Doe}
%	{Master of Science}
%	{Department of Computer Science}
%	{May 2009}
% The \thesistitle definition is in thesis.sty.  Other customizations
% can be made there.
\thesistitle
	{Thesis: \\
	 Alternatives of static type system and it's application for Clojure
programming language.\\
	 Submission to the School of Graduate Studies  \\}
	{\emph{Andrew Tropin}}
	{Bachelor of \emph{Science}}
	{Department of \emph{Computer Science}}
	{\emph{Month Year}}

\addcontentsline{toc}{chapter}{Abstract}
\begin{center}
\textbf{\large Abstract}
\end{center}

This document provides information on how to workaround some drawbacks of
dynamically typed languages. Lack of documentation about function parameters
types. The need to write a lot of tests to check consistency of the project and
integration between components. Implementation hardly rely on
\textbf{clojure.spec} library.

First paragraph: state what the thesis is about, give a simple statement of aims and
methods

Second paragraph: explain the structure of the thesis and say something about the
content

Third paragraph: give a concluding statement, including a short summary of the
results

Additional information about writing abstract can be found here: \\
https://www.th-wildau.de/fileadmin/dokumente/studiengaenge/europaeisches_management/dokumente/Dokumente_EM_Ba/Abstracts_in_English.pdf

\vspace{1cm}

% \emph{``The purpose of the abstract, which should not exceed 150 words for
% a Masters' thesis or 350 words for a Doctoral thesis, is to provide
% sufficient information to allow potential readers to decide on relevance
% of the thesis. Abstracts listed in Dissertation Abstracts International
% or Masters' Abstracts International should contain appropriate key
% words and phrases designed to assist electronic searches.''}

\hfill --- MUN School of Graduate Studies

\include{ack}
\include{contents}
\include{tables}
\include{figures}

\pagenumbering{arabic}
\chapter{Introduction}
\label{chap:intro}

\section{Overview}

There are two main approaches of type checking: static and dynamic. Advocates of
first approach say that benefits of static type checking include earlier
detection of bugs and mistakes (e.g. preventing adding string to an integer),
inline documentation and additional data for advanced code competition, useful
information for compiler, which helps to make optimizations (e.g. replacing
virtual calls by direct calls, when type of the caller are known), improved
runtime efficiency (e.g. not necessary to do dynamic dispatching).

On the other hand, there are also advocates of dynamic type checking who states
that languages with such approach have following advantages: interactive dynamic
development (e.g. REPL driven development + TDD, where tests can be rerun
without recompilation of the whole project), fast prototyping (e.g. type
decisions can be postponed, language are much more expressive and more suitable
for development of systems with rapidly changing or unknown requirements).

Static type checking is a good tool, but it necessary to understand that it
doesn't guarantee programs to be bug-free and correct because it provides
compile-time abstraction over run-time behavior of the software system this
means that some errors can be caught, but execution of the program can still go
wrong. Another, more extensive explanation provided by David Maclver in his work
\cite{staticbroken}. Moreover static typing force to make decisions earlier,
slowdown compilation and make interactive development nearly impossible.

Nowadays many people/teams/companies interested in making data-intensive
systems, just take a look at current trends: data-driven approaches, BigData,
machine learning becomes more and more popular. The reason is clear: huge amount
of data allows to extract useful knowledge, but we need some tools to do it.
Probably dynamism is one of the most important thing for such kind of software
systems because majority of data is not fully structured, according to research
project report of Berkeley University \cite{lymanmuch} around 95\%. In cases when
structure is known up front after few steps people or algorithms generates
queries based on run-time information and data also gets dynamic nature.

Dynamic type checking is a good tool for data-intensive or other kind of dynamic
systems, but it also necessary to understand that it has some trade-offs. It is
much harder to refactor, develop and maintain systems with such type checking
because any even small change can break the system and nobody will know it until
fault will occur later in run-time. There are some techniques, which helps to
deal with it like Test-driven development \cite{beck2003test} or Contract-driven
development \cite{meyer2007contract}, but they doesn't solve all the problems.
Sometimes it is pretty handy to define shape of the date up front to check if
current data conform the specification or to generate data samples from such
specifications.

Information about when and what type checking better to use can be found in
\cite{meijer2004static}, it states that use static typing where possible and
dynamic when needed, but what if there is no possibility to choose type checking
system and it is necessary to use dynamically typed language? Can benefits of
static type checking be achieved in such environment?


\section{Objectives}
In this thesis introduced few ways to get in dynamically typed language
alternatives of benefits, which provides static type checking. As a basis
\textbf{Clojure} programming language was taken, but most of the work should be
suitable for languages, which have same or similar properties described in Chapter
\ref{chap:background}.

Initially, described benefits of static type checking and which of them are more
in demand. After that explained ideas, which help to get similar benefits in
dynamically typed language. Also, some of the existing tools and approaches,
which implements this ideas are explained.

More precisely, following points covered:
\begin{itemize}
\item Improved developer experience
\item Better communication
\item More robust software
\end{itemize}

Improved developer experience means that it is static type system in some
languages provides good and clean error messages and it is good to have tools,
which helps to understand a problem and find its location quicker and with less
effort. Moreover, it is important for hosted language to have tool, which
provides consistent error messages across platforms instead of native stack
traces.

Better communication means that dynamically typed languages has a lack of type
annotations and it is hard to understand what functions consumes as parameters
and what produces. As mentioned in \cite{janes2014lean} documentation often
became outdated and useless, that is why solution must be part of codebase for
more leverage and power and be up-to-date (actual for current codebase).

More robust software means that static type checking provides capabilities for
static analysis, which helps to check kind of soundness between components.
In dynamically typed languages it often hard to make static analysis of the
system, but with gradual typing it is possible \cite{tobin2008design}. Another
important tool for creation of robust software is tests and with type
annotations it is possible to generate bunch of them automatically.

This thesis work provides explanation how to achieve points mentioned above
using some kind of optional annotations without breaking existing workflows
(REPL-driven or/and test-driven development approaches), but it is important to
understand that it is always necessary to do trade offs for each solution, that
is why in Chapter \ref{chap:evaluation} provided an explanation of the cost of
such solution.
% On top of that, the example of the tool, which shows how to automate the process
% of generating tests and enriching documentation string in run-time is introduced
% as a part of contribution of this thesis. This tool provides an interface
% (functions), which allows to modify meta-data of functions in run-time
% namespace-wide or project-wide and another interface (macros), which allows to
% generate test for annotated function in a single namespace.



% \section{Outline}
% Chapter \ref{chap:background} provides an overview of benefits, existing tools
% and possible alternatives






% % \section{Draft section}
% This is the introductory chapter.  This will give you some
% % ideas on how to use \LaTeX~\cite{lam1994} to typeset your document.
% Here is a sample quote using the \verb+\munquote+ environment:

% % \begin{munquote}[~\cite{lam1994}]%
% % \LaTeX{} is a system for typesetting documents.  Its first widely
% % available version, mysteriously numbered 2.09, appeared in 1985.  \LaTeX{}
% % is now extremely popular in the scientific and academic communities, and
% % it is used extensively in industry.  It has become a \emph{lingua franca}
% % of the scientific world; scientists send their papers electronically to
% % colleagues around the world in the form of \LaTeX{} input.%
% % \end{munquote}

% The citation at the end is optional --- if you don't need it,
% then use \verb+\munquote+ without any arguments:

% \begin{munquote}%
% Here is a quote that does not have an associated citation
% after it.  You can specify the citation before or after the
% quote manually.%
% \end{munquote}

% By default, all text is double spaced, however, quotes and footnotes
% must be singled spaced.\munfootnote{This is a single spaced footnote.
% SGS requires that footnotes be singled spaced and this can be done with
% the \texttt{$\backslash$munfootnote} command.} The left margin is slightly
% wider than the right margin.  This is to compensate for binding.

% An example mathematical formulae is show in
% Equation~\ref{eqn:sum}.

% \begin{muneqn}{sum}
% \sum_{i = 0}^{n} i^2
% \end{muneqn}

% A slightly more complicated equation is given in Equation~\ref{eqn:schrodinger}:
% \munfootnote{Equation taken from the \textsl{Schr\"{o}dinger equation}
% entry on \textsl{Wikipedia}}

% \begin{muneqn}{schrodinger}
% i\hbar \frac{\partial}{\partial t}\Psi(x,\,t)=
% -\frac{\hbar^2}{2m}\nabla^2\Psi(x,\,t) + V(x)\Psi(x,\,t)
% \end{muneqn}

% % \section{Cross References}
% \label{sec:xrefs}

% In addition to using \verb+\ref+ to refer to equations, you can also use
% it (in conjunction with the \verb+\label+ command) to refer to sections
% and chapters without hard coding the numbers themselves.  For example,
% this is Section~\ref{sec:xrefs} of Chapter~\ref{chap:intro}.  You can
% also refer to Appendix~\ref{apdx:somelabel}, Subsection~\ref{sec:nested}
% below or any other place that has a \verb+\label+.  You can also use
% labels to refer to a page.  For example, Chapter~\ref{chap:figtab}
% starts on page~\pageref{chap:figtab}.

% % \section{Some Suggestions}

% Here are a few recommendations:

% \begin{itemize}
% 	\item Before using this template, make sure you check with
% 		your supervisor.
% 	\item MUN's library provides electronic access to some \LaTeX{}
% 		related textbooks which can be read online.  Use
% 		the search term \texttt{latex (computer file)} on the
% 		Library's web page.
% 	\item If you run into a problem, Google may be a helpful resource.
% 	\item Concentrate on content, let \LaTeX{} handle the typesetting.
% 	\item Don't worry about warnings related to:
% 	\begin{itemize}
% 		\item overfull \texttt{hboxes}/\texttt{boxes}
% 		\item underfull \texttt{hboxes}/\texttt{vboxes}
% 	\end{itemize}
% 	These can be corrected with modest rewording of your text prior
% 	to submission of your final copy.
% \end{itemize}

% % \section{The \texttt{Makefile}}

% You can use \texttt{make} to ``build'' your thesis on the Linux command
% line\munfootnote{Linux is available on all machines running LabNet in
% \textsl{The Commons} and in other computer labs on campus.} This will
% automatically run the \texttt{bibtex} program to create your bibliography
% and will also re-run \texttt{latex} as necessary to ensure that all
% references are resolved.  A device independent file (\texttt{thesis.dvi})
% will be created, by default.  If you are using this template in another
% environment other than the Linux command line, then the \texttt{Makefile}
% will probably not be useful to you.

% \begin{itemize}
% \item To make a PostScript copy of your thesis, type the following
% at the command line:

% \texttt{make thesis.ps}

% \item To generate a PDF copy of your thesis, run:

% \texttt{make thesis.pdf}

% \item To generate a PDF/A-1b copy of your thesis (which should
% satisfy the SGS's ethesis submission requirements):

% \texttt{make ethesis.pdf}

% \item To remove all the files generated by \texttt{bibtex} and
% \texttt{latex}, use the command:

% \texttt{make clean}

% \item To remove the intermediate files, but leave the PostScript
% and DVI/PDF files intact, use the command:

% \texttt{make neat}
% \end{itemize}

% As you add or remove figures, chapters, or appendices to your thesis,
% make sure you keep the \texttt{Makefile} upto date, too (see the
% \texttt{FIGURES} and \texttt{FILES} macros in the \texttt{Makefile}).

% % \section{Changing Fonts}

% Change fonts: {\Large Large},
% \verb+verbatim ~@#$%^&*(){}[]+,
% \textsc{Small Caps},
% \textsl{slanted text},
% \emph{emphasized text},
% \texttt{typewriter text}.

% % \section{Accents and Ligatures}

% Some accents:
% \'{e}
% \`{e}
% \^{o}
% \"{u}
% \c{c}
% \"{\i}
% \'{\i}
% \~{n}
% \={a}
% \v{a}
% \u{a}

% \noindent Some ligatures:
% fl{\ae}ffi

\chapter{Figures and Tables}
\label{chap:figtab}

\section{Figures}

We can include encapsulated PostScript\texttrademark\ figures
(\texttt{.eps}) in the document and refer to it using a label.
For example, MUN's logo can be seen in Figure~\ref{fig:MUN_Logo_Pantone}.
\munepsfig{MUN_Logo_Pantone}{This is MUN's logo}

Figure~\ref{fig:enrollment} shows a chart of MUN's Fall
enrollment from 2005 -- 2009.\munfootnote{From \emph{Memorial
University of Newfoundland --- Fact Book 2009}.}
\munepsfig[scale=0.50]{enrollment}{MUN Fall Enrollment 2005 -- 2009}
The figure was created using the \textsf{Calc} spreadsheet application of
the office suite \textsf{OpenOffice.org}.\munfootnote{This office suite
can be downloaded at no cost from \texttt{http://openoffice.org/}. Unlike
other commercial office suites, \textsf{OpenOffice.org} may be legally
shared with colleagues and fellow students.  There are versions for
Linux, Microsoft Windows, Mac~OS~X and Solaris.  Also, unlike commercial
offerings, \textsf{OpenOffice.org} does not require activation using
registration keys.}  This figure was reduced by 50\%.

For larger figures, we can use landscape mode to rotate the page
and display the figure using the \verb+\munlepsfig+ command, as shown
in Figure~\ref{fig:enrollment-landscape}.  The figure will be the
only thing on the page when typeset in landscape mode. 
(The figure is reduced to 85\% of its original size.)
\munlepsfig[scale=0.85]{enrollment-landscape}
	{MUN Fall Enrollment 2005 -- 2009 (landscape)}

Alternatively, if we just want to rotate the figure, but not 
the entire page, we can specify an \texttt{angle} attribute
in the default argument of the \verb+\munepsfig+ command.
The result is shown in Figure~\ref{fig:enrollment-rotate}.
If the figure is too large or if there isn't sufficient
text, then the figure may appear on its own page.
\munepsfig[scale=0.30,angle=90]{enrollment-rotate}
	{MUN Fall Enrollment 2005 -- 2009 (rotated)}

Note that all three of the enrollment figures are basically
the same file, but with different names --- on Linux, they are
symbolic links to the same file.  The filenames have to be different
because the reference labels need to be unique.

Figure~\ref{fig:db-deadlock} shows a Petri net created using the
\texttt{xfig} program (\texttt{http://www.xfig.org/}) which has
very good support for \LaTeX.  This figure has been
reduced to 40\% of its original size.
\munepsfig[scale=0.40]{db-deadlock}{A deadlocked Petri net}

We can also create figures of text (such as short code snippets)
using the \verb+\muntxtfig+ command, as show in Figure~\ref{fig:code}.
\begin{muntxtfig}[1.0]{code}{Hello World}{0.5\textwidth}
\begin{verbatim}
#include <stdio.h>

int main(int argc, char **argv)
{
  printf("Hello world!\n");
  exit(0);
}
\end{verbatim}
\end{muntxtfig}

\section{Tables}

We can also create tables, as seen by Table~\ref{tab:pop}.  Note that,
as required by SGS guidelines, the caption for a table appears above the
table whereas figure captions appear below the figures.  Tables and
figures can ``float'' --- they may not appear on the page on which they
are mentioned.  \LaTeX{} tries to handle figure and table placement
intelligently, but if if you have a lot of them without a reasonable
amount of surrounding textual content, the figures and tables can
accumulate towards the end of the chapter.  Generally speaking, if
there is sufficient text explaining the tables and figures or if the
tables/figures are relatively small, this may not be a problem.  However,
if you have a lot of tables or figures, it may be a good idea to put
them in an appendix and refer to them as the need arises.

\begin{muntab}{c||c|c|c||c|c|c|}{pop}{Fall Semester Enrollment}
\hline
	& \multicolumn{3}{c||}{Undergraduate}
	& \multicolumn{3}{c|}{Graduate} \\
\cline{2-7}
     & F/T & P/T & Total & F/T & P/T & Total \\
\cline{2-7}
2004 & 13,191 & 2,223 & 15,414 & 1,308 & 879 & 2,187 \\
2005 & 13,184 & 2,143 & 15,327 & 1,375 & 920 & 2,295 \\
2006 & 12,809 & 2,224 & 15,033 & 1,373 & 899 & 2,272 \\
2007 & 12,634 & 2,155 & 14,789 & 1,403 & 899 & 2,302 \\
2008 & 12,269 & 2,208 & 14,477 & 1,410 &1,005& 2,415 \\
2009 & 12,382 & 2,323 & 14,705 & 1,567 &1,106& 2,673 \\
\hline
\end{muntab}

Table~\ref{tab:degrees} shows a different table in landscape
mode.\munfootnote{This data was also taken from the \emph{Memorial
University of Newfoundland --- Fact Book 2009}.} This is useful if your
table is too wide for the page.  Tables are double-spaced by default.
To single-space a table, change the \verb+\baselinestretch+ before
beginning the table environment.  Remember to restore it after the
environment has ended.

\renewcommand{\baselinestretch}{1.0}\normalsize
\begin{munltab}{lrrrrrrrrrrrr}
	{degrees}
	{Masters Degrees Conferred by Convocation Session --- 1950 to 2009}
\cline{2-13}
				&
\multicolumn{2}{|c|}{2009}	&
\multicolumn{2}{c|}{2008}	& 
\multicolumn{2}{c|}{2007}	& 
\multicolumn{2}{c|}{2006}	& 
\multicolumn{2}{c|}{2006}	& 
\multicolumn{1}{c|}{1950--2004}	& 
\multicolumn{1}{c|}{Total}	\\
\cline{2-13}
	  &
May & Oct &
May & Oct &
May & Oct &
May & Oct &
May & Oct & &  \\
Degrees \\
\hline
Master of Applied Science		&  14 &   2 &  15 &   8 &  28 &   1 &  21 &   3 &   3 &   1 &    98 &   194 \\
Master of Applied Social Psychology     &   1 &   5 &   2 &   5 &   1 &   4 &   0 &   4 &   0 &   4 &    28 &    54 \\
Master of Applied Statistics            &   0 &   0 &   3 &   1 &   0 &   0 &   1 &   0 &   0 &   0 &    19 &    24 \\
Master of Arts                          &  37 &  49 &  26 &  43 &  14 &  42 &  14 &  56 &  13 &  44 &   994 & 1,332 \\
Master of Business Administration       &  14 &  16 &  23 &   6 &  33 &  12 &  33 &  11 &  33 &   8 &   818 & 1,007 \\
Master of Education                     & 107 &  87 & 120 &  55 & 147 &  74 & 108 &  76 & 113 &  75 & 2,603 & 3,565 \\
Master of Employment Relations          &   8 &   9 &   5 &   7 &   7 &  14 &   4 &   9 &   3 &   5 &    12 &    83 \\
Master of Engineering                   &  20 &  19 &  20 &  10 &  16 &  10 &  15 &  13 &   4 &  19 &   440 &   586 \\
Master of Environmental Science         &   3 &   3 &   3 &   1 &   0 &   1 &   7 &   1 &   3 &   1 &    66 &    89 \\
Master of Marine Studies                &   2 &   0 &   0 &   1 &   0 &   2 &   2 &   2 &   1 &   2 &    26 &    38 \\
Master of Music                         &   4 &   1 &   5 &   0 &   3 &   0 &   3 &   0 &   3 &   0 &     7 &    26 \\
Master of Nursing                       &   7 &   8 &  10 &   4 &  17 &   4 &  23 &   7 &   6 &   1 &   116 &   203 \\
Master of Oil and Gas Studies           &   0 &   0 &   2 &   0 &   0 &   0 &   0 &   2 &   4 &   0 &     0 &     8 \\
Master of Philosophy                    &   5 &   4 &   2 &   1 &   5 &   2 &   5 &   3 &   2 &   0 &   112 &   141 \\
Master of Physical Education            &   0 &   2 &   3 &   0 &   5 &   4 &   3 &   0 &   4 &   4 &    84 &   109 \\
Master of Public Health                 &   0 &   8 &   0 &   0 &   0 &   0 &   0 &   0 &   0 &   0 &     0 &     8 \\
Master of Science                       &  40 &  32 &  41 &  19 &  29 &  25 &  35 &  29 &  32 &  23 & 1,653 & 1,958 \\
Master of Science (Kinesiology)         &   1 &   0 &   4 &   2 &   1 &   2 &   2 &   6 &   4 &   3 &     0 &    25 \\
Master of Science (Medicine)            &  18 &   7 &  11 &   8 &  10 &   5 &   9 &   9 &   8 &   4 &     0 &    89 \\
Master of Science (Pharmacy)            &   0 &   0 &   1 &   1 &   0 &   0 &   0 &   0 &   1 &   0 &    16 &    19 \\
Master of Social Work                   &   4 &  11 &   4 &   5 &   4 &   9 &   9 &   5 &   4 &  10 &   257 &   322 \\
Master of Women's Studies               &   2 &   0 &   2 &   0 &   1 &   1 &   2 &   3 &   2 &   0 &    20 &    33 \\
\hline
\textbf{Total Masters}                  & 287 & 263 & 302 & 177 & 321 & 212 & 296 & 239 & 243 & 204 & 7,369 & 9,913 \\
\end{munltab}
\renewcommand{\baselinestretch}{\spacing}\normalsize

\chapter{Dealing with Errors}
\label{chap:errors}

\LaTeX{} can produce cryptic error messages at times.
However, with some experience, it is usually not too
difficult to determine what the problem is and how to fix it.

As mentioned earlier, appropriate search terms in Google
may help you fix these error messages.


\chapter{Lorem Ipsum}
\label{chap:ch4_abbr}
Now, for your reading pleasure, some \textsl{Lorem ipsum}, courtesy
of:
\begin{center}
\texttt{<http://www.lipsum.com/>}
\end{center}
This gives a good view of the margins --- note that the left margin
is a bit wider than the right margin to accommodate binding.

Lorem ipsum dolor sit amet, consectetur adipiscing elit. Etiam odio elit,
viverra eu tempor non, pulvinar ac nisi. Pellentesque habitant morbi
tristique senectus et netus et malesuada fames ac turpis egestas. Sed
adipiscing, dui quis viverra facilisis, quam libero adipiscing justo,
vitae dictum libero mauris ac magna. Aenean sem ligula, vulputate at
vestibulum eu, pellentesque in justo. Sed et eros mauris, sed placerat
nulla. Maecenas nulla velit, facilisis et rutrum nec, volutpat id
lorem. Duis vestibulum odio velit, id elementum tortor. Sed pellentesque
leo ac nibh iaculis at fermentum orci lobortis. Suspendisse arcu magna,
porta nec pretium non, feugiat vitae orci. Vivamus at enim arcu,
at sagittis nisl. Vestibulum at mi enim, vel malesuada justo. Class
aptent taciti sociosqu ad litora torquent per conubia nostra, per
inceptos himenaeos. Nullam sed nunc at enim posuere sagittis. Vivamus
augue turpis, mattis a blandit non, sollicitudin non nisl. Integer
vestibulum, est vitae cursus adipiscing, elit libero pretium leo,
in scelerisque augue felis volutpat nisl. Donec commodo posuere arcu,
eget feugiat dui ornare nec. Nullam eros mi, condimentum ac ultricies ac,
euismod lobortis nibh. Cras ac ligula pharetra risus elementum pharetra
vel in quam. Fusce ac augue vulputate nibh imperdiet convallis sit amet
et quam. Integer porttitor dictum fermentum.

Nullam id ante arcu. Nulla facilisi. Vestibulum sodales, mi sodales
ultricies pulvinar, orci leo dictum diam, quis imperdiet turpis lacus
ut sem. Nulla rutrum odio sit amet elit aliquam blandit gravida nunc
placerat. Aenean et neque ut leo condimentum vehicula. Fusce quis orci
vitae enim dapibus tincidunt in vel ipsum. Phasellus auctor neque ac eros
egestas sit amet ultricies erat vestibulum. Ut erat ligula, pharetra
vel hendrerit vitae, mattis ac turpis. Ut malesuada diam vitae lacus
vestibulum a tempus nisl posuere. Ut nisi sem, dictum eu laoreet sed,
commodo eget enim. Morbi vel lacus neque, tempus fringilla tellus. Nunc
id egestas felis. Nullam eu mollis neque. Ut non mauris malesuada
eros sagittis congue. Cras vitae felis ut nisl mollis semper ut quis
risus. Sed eu arcu urna, et commodo sapien. Donec vestibulum, libero
sit amet ultrices blandit, erat lorem volutpat lectus, sed feugiat leo
elit in orci. Aliquam vitae leo tellus, placerat pulvinar massa. Nulla
at sapien hendrerit diam varius vehicula.

Curabitur et orci nulla. Phasellus euismod, massa non hendrerit dictum,
dolor enim imperdiet sapien, vitae commodo lorem tellus eu quam. Duis
egestas felis velit. Sed in orci nec nulla rutrum posuere. Suspendisse
potenti. Nunc vel quam nisi. In at molestie libero. Aenean hendrerit
vestibulum orci, ut hendrerit nulla volutpat lacinia. Vestibulum sit amet
sapien vitae lectus gravida vehicula. Suspendisse ac purus sit amet est
congue auctor.

Morbi pellentesque, quam vel mattis molestie, augue purus vestibulum
lorem, nec consequat enim eros eu augue. In odio dolor, scelerisque
a lobortis porttitor, commodo ut lacus. Maecenas sit amet diam
nec tellus accumsan bibendum. Praesent in turpis velit, malesuada
commodo sapien. Nunc ornare urna enim. Sed at diam non metus porttitor
suscipit. Aliquam erat volutpat. Duis aliquet magna in mauris semper
placerat. Ut eget quam orci. Ut egestas, dolor at dapibus accumsan, leo
nibh egestas urna, ac consectetur dui odio quis eros. Nam libero dolor,
lacinia eget imperdiet non, malesuada vehicula diam. Etiam id ipsum eget
turpis consectetur tristique id at ante. Vivamus blandit nunc eu nisl
varius sed accumsan odio molestie.


\include{chap5}
\chapter{Conclusions}
\label{chap:conclusions}
That's all folks!


\documentclass[a4paper,14pt]{extreport}
\usepackage[utf8]{inputenc}
\usepackage[english,russian]{babel}
% \usepackage[dvips]{graphicx}
\usepackage[pdftex]{graphicx}
\usepackage{geometry}
\geometry{left=25mm}
\geometry{right=20mm}
\geometry{top=20mm}
\geometry{bottom=20mm}
\graphicspath{{image/}}
\usepackage[usenames]{color}
\usepackage{colortbl}
\usepackage{multirow}
\usepackage{amssymb,amsfonts,amsmath,mathtext,cite,enumerate,float}
\usepackage{setspace}
\usepackage{indentfirst}
\onehalfspacing


\begin{document}
\begin{titlepage}
\begin{table}[]
    \centering
    \begin{tabular}{rcl}
    Автономная некоммерческая &
    \multirow{4}{*}{\includegraphics[width=40mm]{image/logo.eps}}
          & Autonomous noncommercial \\
    организация высшего  & & organization of higher \\
    образования & & education \\
    «Университет Иннополис»  &
     & «Innopolis University» \\
    \hline
    \hline
    \end{tabular}
    \label{tab:my_label}
\end{table}
\vline
\vspace{20mm}

\begin{center}
\textbf{АННОТАЦИЯ \\ НА ВЫПУСКНУЮ КВАЛИФИКАЦИОННУЮ РАБОТУ  \\
ПО НАПРАВЛЕНИЮ ПОДГОТОВКИ \\ 09.03.01 --- «ИНФОРМАТИКА И ВЫЧИСЛИТЕЛЬНАЯ ТЕХНИКА»}
\end{center}
\vspace{20mm}


    \begin{tabular}{ll
|>{\columncolor[gray]{.9}}l|}
\cline{3-3}
\textbf{Тема:} &
     &
    \makebox[133mm][l]{Альтернативы статической типизации для Clojure}    \\
    &&\\
    && \\
    &&  \\
\cline{3-3}
    \end{tabular}
\vspace{5mm}


    \begin{tabular}{ll
|>{\columncolor[gray]{.9}}l|l
|>{\columncolor[gray]{.9}}l|}
\cline{3-3} \cline{5-5}
Выполнил &
     &
    \makebox[77mm][l]{Тропин Андрей Геннадьевич}   &
    &    \\
    &&&&
    \makebox[39.5mm]{\textcolor[gray]{.7}{подпись}} \\
    &&&& \\
    \cline{3-3} \cline{5-5}
    \end{tabular}
\vspace{5mm}

\vspace{\fill}

\begin{center}
Иннополис, 2017
\end{center}
\end{titlepage}


\newpage
% \noindent {\large \textbf{Содержание}} \\

\chapter*{Содержание}
\documentclass[dvips,letterpaper,12pt]{report}
\usepackage{thesis}

\begin{document}

\pagenumbering{roman}

% Fill in the title, author, degree name, department, and month/year.
% Upon completion, this should look like the following:
%\thesistitle
%	{Complicated and Important-Sounding Thesis Title}
%	{John P. Doe}
%	{Master of Science}
%	{Department of Computer Science}
%	{May 2009}
% The \thesistitle definition is in thesis.sty.  Other customizations
% can be made there.
\thesistitle
	{Thesis: \\
	 Alternatives of static type system and it's application for Clojure
programming language.\\
	 Submission to the School of Graduate Studies  \\}
	{\emph{Andrew Tropin}}
	{Bachelor of \emph{Science}}
	{Department of \emph{Computer Science}}
	{\emph{Month Year}}

\include{abstract}
\include{ack}
\include{contents}
\include{tables}
\include{figures}

\pagenumbering{arabic}
\include{chap1}
\include{chap2}
\include{chap3}
\include{chap4}
\include{chap5}
\include{chap6}

\include{bib}

% If you have no appendices, remove the following two lines.
% If you have more appdences, add them as necessary.
\appendix
\include{apdxa}

\end{document}


% \newpage
% \noindent {\large \textbf{Введение}}
% \newline
\chapter*{Введение}

Существует два основных вида систем типизации, применяемых в современных языках
программирования: статическая и динамическая. Сторонники первого подхода
говорят, что преимущества статической проверки типов включает в себя: более
раннее обнаружение ошибок (например, предотвращает сложение переменных
строкового и целочисленного типа), дополнительную документацию и метаданные для
прогрессивного автодополнения фрагментов кода средствами интегрированной среды
разработки, а также возможность предоставлять компилятору вспомогательную
информацию, позволяющую производить оптимизации во время компиляции (например,
замена виртуальных вызовов прямыми вызовами функций).

С другой стороны, есть и сторонники динамической проверки типов, которые
считают, что языки с таким подходом имеют следующие преимущества: интерактивный
способ разработки (например, разработка на основе техник с использование REPL +
TDD, где тесты могут быть запущены без перекомпиляции всего проекта), быстрое
прототипирование (например, решения касательно типов могут быть отложены, язык
гораздо более выразителен и больше подходит для разработки систем с быстро
изменяющимися или неизвестными требования).

Статическая проверка типов - хороший инструмент, но нужно понимать, что он не
гарантирует, что программы будут безошибочными и правильными, поскольку он
обеспечивает всего лишь абстракцию во время компиляции над поведением запущенной
программной системы, это значит, что некоторые ошибки могут быть обнаружены, но
выполнение программы всё же может пойти не так. Другое, более обширное
объяснение, представленное Дэвидом Маклвером в его работе [11] более подробно
обозначает причины по которым невозможно предотвратить все ошибки во время
компиляции. Кроме того, статическая типизации заставляет разрбаотчика принимать
решения раньше, замедляет процесс компиляции и делает интерактивную разработку
практически невозможной.

В настоящее время многие люди / команды / компании заинтересованы в создании
систем оперирующих с большими объёмами данных, просто взгляните на текущие
тенденции: BigData, машинное обучение становятся все более популярными. Причина
ясна: огромные количество данных позволяет извлечь полезные знания, но для этого
нужны подходящие инструменты. Вероятно, динамичность является одним из наиболее
важных свойств языка для такого рода программных систем, поскольку большинство
данных не полностью структурировано, согласно исследовательскому отчету по
проекту Университета Беркли [10] около $95\%$. В случаях когда структура
известна заранее по прошествии нескольких шагов, люди или алгоритмы генерируют
запросы, основанные на информации и данных времени выполнения, также приобретают
динамический характер.

Динамическая проверка типов - хороший инструмент для интенсивно использующих
данные или других типов динамических систем, но важно понимать недостатки такого
подхода. Как правило приозводить рефакторинг, разрабатывать и поддерживать
системы, использующие языки с динамической типизацией сложнее, так как любое
даже небольшое изменение может в перспективе привести к краху системы, и никто
не узнает об этом, пока ошибка не возникнет позже во время выполнения. Есть
несколько техник, которые помогают справиться с проблемами такого рода, например
разработка через тестирование [1] или разработка на основе контрактов [14], но
они не решают всех проблем. Иногда удобно декларировать форму данных с помощью
спецификаций, чтобы впоследствии проверять, соответствуют ли текущие данные
формату или создавать образцы данных из таких спецификаций.

Информацию о том, когда и какой вид системы типизации лучше использовать, можно
найти в статье [13], авторы говорят, что по возможности нужно использовать
статическую типизацию и только, если необходимо - динамическую, но что, если нет
возможности выбрать систему проверки типов и необходимо использовать динамически
типизированный язык? Могут ли быть преимущества статической проверки типов
получены в таком окружении?

\chapter*{Основная часть}
% \chapter*{Заключени}



\documentclass[dvips,letterpaper,12pt]{report}
\usepackage{thesis}

\begin{document}

\pagenumbering{roman}

% Fill in the title, author, degree name, department, and month/year.
% Upon completion, this should look like the following:
%\thesistitle
%	{Complicated and Important-Sounding Thesis Title}
%	{John P. Doe}
%	{Master of Science}
%	{Department of Computer Science}
%	{May 2009}
% The \thesistitle definition is in thesis.sty.  Other customizations
% can be made there.
\thesistitle
	{Thesis: \\
	 Alternatives of static type system and it's application for Clojure
programming language.\\
	 Submission to the School of Graduate Studies  \\}
	{\emph{Andrew Tropin}}
	{Bachelor of \emph{Science}}
	{Department of \emph{Computer Science}}
	{\emph{Month Year}}

\include{abstract}
\include{ack}
\include{contents}
\include{tables}
\include{figures}

\pagenumbering{arabic}
\include{chap1}
\include{chap2}
\include{chap3}
\include{chap4}
\include{chap5}
\include{chap6}

\include{bib}

% If you have no appendices, remove the following two lines.
% If you have more appdences, add them as necessary.
\appendix
\include{apdxa}

\end{document}

\end{document}


% If you have no appendices, remove the following two lines.
% If you have more appdences, add them as necessary.
\appendix
\chapter{Appendix title}
\label{apdx:somelabel}
This is Appendix~\ref{apdx:somelabel}.

You can have additional appendices too
(\emph{e.g.}, \texttt{apdxb.tex}, \texttt{apdxc.tex}, \emph{etc.}).
If you don't need any appendices, delete the appendix
related lines from \texttt{thesis.tex} and the file names
from \texttt{Makefile}.


\end{document}


% \newpage
% \noindent {\large \textbf{Введение}}
% \newline
\chapter*{Введение}

Существует два основных вида систем типизации, применяемых в современных языках
программирования: статическая и динамическая. Сторонники первого подхода
говорят, что преимущества статической проверки типов включает в себя: более
раннее обнаружение ошибок (например, предотвращает сложение переменных
строкового и целочисленного типа), дополнительную документацию и метаданные для
прогрессивного автодополнения фрагментов кода средствами интегрированной среды
разработки, а также возможность предоставлять компилятору вспомогательную
информацию, позволяющую производить оптимизации во время компиляции (например,
замена виртуальных вызовов прямыми вызовами функций).

С другой стороны, есть и сторонники динамической проверки типов, которые
считают, что языки с таким подходом имеют следующие преимущества: интерактивный
способ разработки (например, разработка на основе техник с использование REPL +
TDD, где тесты могут быть запущены без перекомпиляции всего проекта), быстрое
прототипирование (например, решения касательно типов могут быть отложены, язык
гораздо более выразителен и больше подходит для разработки систем с быстро
изменяющимися или неизвестными требования).

Статическая проверка типов - хороший инструмент, но нужно понимать, что он не
гарантирует, что программы будут безошибочными и правильными, поскольку он
обеспечивает всего лишь абстракцию во время компиляции над поведением запущенной
программной системы, это значит, что некоторые ошибки могут быть обнаружены, но
выполнение программы всё же может пойти не так. Другое, более обширное
объяснение, представленное Дэвидом Маклвером в его работе [11] более подробно
обозначает причины по которым невозможно предотвратить все ошибки во время
компиляции. Кроме того, статическая типизации заставляет разрбаотчика принимать
решения раньше, замедляет процесс компиляции и делает интерактивную разработку
практически невозможной.

В настоящее время многие люди / команды / компании заинтересованы в создании
систем оперирующих с большими объёмами данных, просто взгляните на текущие
тенденции: BigData, машинное обучение становятся все более популярными. Причина
ясна: огромные количество данных позволяет извлечь полезные знания, но для этого
нужны подходящие инструменты. Вероятно, динамичность является одним из наиболее
важных свойств языка для такого рода программных систем, поскольку большинство
данных не полностью структурировано, согласно исследовательскому отчету по
проекту Университета Беркли [10] около $95\%$. В случаях когда структура
известна заранее по прошествии нескольких шагов, люди или алгоритмы генерируют
запросы, основанные на информации и данных времени выполнения, также приобретают
динамический характер.

Динамическая проверка типов - хороший инструмент для интенсивно использующих
данные или других типов динамических систем, но важно понимать недостатки такого
подхода. Как правило приозводить рефакторинг, разрабатывать и поддерживать
системы, использующие языки с динамической типизацией сложнее, так как любое
даже небольшое изменение может в перспективе привести к краху системы, и никто
не узнает об этом, пока ошибка не возникнет позже во время выполнения. Есть
несколько техник, которые помогают справиться с проблемами такого рода, например
разработка через тестирование [1] или разработка на основе контрактов [14], но
они не решают всех проблем. Иногда удобно декларировать форму данных с помощью
спецификаций, чтобы впоследствии проверять, соответствуют ли текущие данные
формату или создавать образцы данных из таких спецификаций.

Информацию о том, когда и какой вид системы типизации лучше использовать, можно
найти в статье [13], авторы говорят, что по возможности нужно использовать
статическую типизацию и только, если необходимо - динамическую, но что, если нет
возможности выбрать систему проверки типов и необходимо использовать динамически
типизированный язык? Могут ли быть преимущества статической проверки типов
получены в таком окружении?

\chapter*{Основная часть}
% \chapter*{Заключени}



\documentclass[dvips,letterpaper,12pt]{report}
\usepackage{thesis}

\begin{document}

\pagenumbering{roman}

% Fill in the title, author, degree name, department, and month/year.
% Upon completion, this should look like the following:
%\thesistitle
%	{Complicated and Important-Sounding Thesis Title}
%	{John P. Doe}
%	{Master of Science}
%	{Department of Computer Science}
%	{May 2009}
% The \thesistitle definition is in thesis.sty.  Other customizations
% can be made there.
\thesistitle
	{Thesis: \\
	 Alternatives of static type system and it's application for Clojure
programming language.\\
	 Submission to the School of Graduate Studies  \\}
	{\emph{Andrew Tropin}}
	{Bachelor of \emph{Science}}
	{Department of \emph{Computer Science}}
	{\emph{Month Year}}

\addcontentsline{toc}{chapter}{Abstract}
\begin{center}
\textbf{\large Abstract}
\end{center}

This document provides information on how to workaround some drawbacks of
dynamically typed languages. Lack of documentation about function parameters
types. The need to write a lot of tests to check consistency of the project and
integration between components. Implementation hardly rely on
\textbf{clojure.spec} library.

First paragraph: state what the thesis is about, give a simple statement of aims and
methods

Second paragraph: explain the structure of the thesis and say something about the
content

Third paragraph: give a concluding statement, including a short summary of the
results

Additional information about writing abstract can be found here: \\
https://www.th-wildau.de/fileadmin/dokumente/studiengaenge/europaeisches_management/dokumente/Dokumente_EM_Ba/Abstracts_in_English.pdf

\vspace{1cm}

% \emph{``The purpose of the abstract, which should not exceed 150 words for
% a Masters' thesis or 350 words for a Doctoral thesis, is to provide
% sufficient information to allow potential readers to decide on relevance
% of the thesis. Abstracts listed in Dissertation Abstracts International
% or Masters' Abstracts International should contain appropriate key
% words and phrases designed to assist electronic searches.''}

\hfill --- MUN School of Graduate Studies

\include{ack}
\include{contents}
\include{tables}
\include{figures}

\pagenumbering{arabic}
\chapter{Introduction}
\label{chap:intro}

\section{Overview}

There are two main approaches of type checking: static and dynamic. Advocates of
first approach say that benefits of static type checking include earlier
detection of bugs and mistakes (e.g. preventing adding string to an integer),
inline documentation and additional data for advanced code competition, useful
information for compiler, which helps to make optimizations (e.g. replacing
virtual calls by direct calls, when type of the caller are known), improved
runtime efficiency (e.g. not necessary to do dynamic dispatching).

On the other hand, there are also advocates of dynamic type checking who states
that languages with such approach have following advantages: interactive dynamic
development (e.g. REPL driven development + TDD, where tests can be rerun
without recompilation of the whole project), fast prototyping (e.g. type
decisions can be postponed, language are much more expressive and more suitable
for development of systems with rapidly changing or unknown requirements).

Static type checking is a good tool, but it necessary to understand that it
doesn't guarantee programs to be bug-free and correct because it provides
compile-time abstraction over run-time behavior of the software system this
means that some errors can be caught, but execution of the program can still go
wrong. Another, more extensive explanation provided by David Maclver in his work
\cite{staticbroken}. Moreover static typing force to make decisions earlier,
slowdown compilation and make interactive development nearly impossible.

Nowadays many people/teams/companies interested in making data-intensive
systems, just take a look at current trends: data-driven approaches, BigData,
machine learning becomes more and more popular. The reason is clear: huge amount
of data allows to extract useful knowledge, but we need some tools to do it.
Probably dynamism is one of the most important thing for such kind of software
systems because majority of data is not fully structured, according to research
project report of Berkeley University \cite{lymanmuch} around 95\%. In cases when
structure is known up front after few steps people or algorithms generates
queries based on run-time information and data also gets dynamic nature.

Dynamic type checking is a good tool for data-intensive or other kind of dynamic
systems, but it also necessary to understand that it has some trade-offs. It is
much harder to refactor, develop and maintain systems with such type checking
because any even small change can break the system and nobody will know it until
fault will occur later in run-time. There are some techniques, which helps to
deal with it like Test-driven development \cite{beck2003test} or Contract-driven
development \cite{meyer2007contract}, but they doesn't solve all the problems.
Sometimes it is pretty handy to define shape of the date up front to check if
current data conform the specification or to generate data samples from such
specifications.

Information about when and what type checking better to use can be found in
\cite{meijer2004static}, it states that use static typing where possible and
dynamic when needed, but what if there is no possibility to choose type checking
system and it is necessary to use dynamically typed language? Can benefits of
static type checking be achieved in such environment?


\section{Objectives}
In this thesis introduced few ways to get in dynamically typed language
alternatives of benefits, which provides static type checking. As a basis
\textbf{Clojure} programming language was taken, but most of the work should be
suitable for languages, which have same or similar properties described in Chapter
\ref{chap:background}.

Initially, described benefits of static type checking and which of them are more
in demand. After that explained ideas, which help to get similar benefits in
dynamically typed language. Also, some of the existing tools and approaches,
which implements this ideas are explained.

More precisely, following points covered:
\begin{itemize}
\item Improved developer experience
\item Better communication
\item More robust software
\end{itemize}

Improved developer experience means that it is static type system in some
languages provides good and clean error messages and it is good to have tools,
which helps to understand a problem and find its location quicker and with less
effort. Moreover, it is important for hosted language to have tool, which
provides consistent error messages across platforms instead of native stack
traces.

Better communication means that dynamically typed languages has a lack of type
annotations and it is hard to understand what functions consumes as parameters
and what produces. As mentioned in \cite{janes2014lean} documentation often
became outdated and useless, that is why solution must be part of codebase for
more leverage and power and be up-to-date (actual for current codebase).

More robust software means that static type checking provides capabilities for
static analysis, which helps to check kind of soundness between components.
In dynamically typed languages it often hard to make static analysis of the
system, but with gradual typing it is possible \cite{tobin2008design}. Another
important tool for creation of robust software is tests and with type
annotations it is possible to generate bunch of them automatically.

This thesis work provides explanation how to achieve points mentioned above
using some kind of optional annotations without breaking existing workflows
(REPL-driven or/and test-driven development approaches), but it is important to
understand that it is always necessary to do trade offs for each solution, that
is why in Chapter \ref{chap:evaluation} provided an explanation of the cost of
such solution.
% On top of that, the example of the tool, which shows how to automate the process
% of generating tests and enriching documentation string in run-time is introduced
% as a part of contribution of this thesis. This tool provides an interface
% (functions), which allows to modify meta-data of functions in run-time
% namespace-wide or project-wide and another interface (macros), which allows to
% generate test for annotated function in a single namespace.



% \section{Outline}
% Chapter \ref{chap:background} provides an overview of benefits, existing tools
% and possible alternatives






% % \section{Draft section}
% This is the introductory chapter.  This will give you some
% % ideas on how to use \LaTeX~\cite{lam1994} to typeset your document.
% Here is a sample quote using the \verb+\munquote+ environment:

% % \begin{munquote}[~\cite{lam1994}]%
% % \LaTeX{} is a system for typesetting documents.  Its first widely
% % available version, mysteriously numbered 2.09, appeared in 1985.  \LaTeX{}
% % is now extremely popular in the scientific and academic communities, and
% % it is used extensively in industry.  It has become a \emph{lingua franca}
% % of the scientific world; scientists send their papers electronically to
% % colleagues around the world in the form of \LaTeX{} input.%
% % \end{munquote}

% The citation at the end is optional --- if you don't need it,
% then use \verb+\munquote+ without any arguments:

% \begin{munquote}%
% Here is a quote that does not have an associated citation
% after it.  You can specify the citation before or after the
% quote manually.%
% \end{munquote}

% By default, all text is double spaced, however, quotes and footnotes
% must be singled spaced.\munfootnote{This is a single spaced footnote.
% SGS requires that footnotes be singled spaced and this can be done with
% the \texttt{$\backslash$munfootnote} command.} The left margin is slightly
% wider than the right margin.  This is to compensate for binding.

% An example mathematical formulae is show in
% Equation~\ref{eqn:sum}.

% \begin{muneqn}{sum}
% \sum_{i = 0}^{n} i^2
% \end{muneqn}

% A slightly more complicated equation is given in Equation~\ref{eqn:schrodinger}:
% \munfootnote{Equation taken from the \textsl{Schr\"{o}dinger equation}
% entry on \textsl{Wikipedia}}

% \begin{muneqn}{schrodinger}
% i\hbar \frac{\partial}{\partial t}\Psi(x,\,t)=
% -\frac{\hbar^2}{2m}\nabla^2\Psi(x,\,t) + V(x)\Psi(x,\,t)
% \end{muneqn}

% % \section{Cross References}
% \label{sec:xrefs}

% In addition to using \verb+\ref+ to refer to equations, you can also use
% it (in conjunction with the \verb+\label+ command) to refer to sections
% and chapters without hard coding the numbers themselves.  For example,
% this is Section~\ref{sec:xrefs} of Chapter~\ref{chap:intro}.  You can
% also refer to Appendix~\ref{apdx:somelabel}, Subsection~\ref{sec:nested}
% below or any other place that has a \verb+\label+.  You can also use
% labels to refer to a page.  For example, Chapter~\ref{chap:figtab}
% starts on page~\pageref{chap:figtab}.

% % \section{Some Suggestions}

% Here are a few recommendations:

% \begin{itemize}
% 	\item Before using this template, make sure you check with
% 		your supervisor.
% 	\item MUN's library provides electronic access to some \LaTeX{}
% 		related textbooks which can be read online.  Use
% 		the search term \texttt{latex (computer file)} on the
% 		Library's web page.
% 	\item If you run into a problem, Google may be a helpful resource.
% 	\item Concentrate on content, let \LaTeX{} handle the typesetting.
% 	\item Don't worry about warnings related to:
% 	\begin{itemize}
% 		\item overfull \texttt{hboxes}/\texttt{boxes}
% 		\item underfull \texttt{hboxes}/\texttt{vboxes}
% 	\end{itemize}
% 	These can be corrected with modest rewording of your text prior
% 	to submission of your final copy.
% \end{itemize}

% % \section{The \texttt{Makefile}}

% You can use \texttt{make} to ``build'' your thesis on the Linux command
% line\munfootnote{Linux is available on all machines running LabNet in
% \textsl{The Commons} and in other computer labs on campus.} This will
% automatically run the \texttt{bibtex} program to create your bibliography
% and will also re-run \texttt{latex} as necessary to ensure that all
% references are resolved.  A device independent file (\texttt{thesis.dvi})
% will be created, by default.  If you are using this template in another
% environment other than the Linux command line, then the \texttt{Makefile}
% will probably not be useful to you.

% \begin{itemize}
% \item To make a PostScript copy of your thesis, type the following
% at the command line:

% \texttt{make thesis.ps}

% \item To generate a PDF copy of your thesis, run:

% \texttt{make thesis.pdf}

% \item To generate a PDF/A-1b copy of your thesis (which should
% satisfy the SGS's ethesis submission requirements):

% \texttt{make ethesis.pdf}

% \item To remove all the files generated by \texttt{bibtex} and
% \texttt{latex}, use the command:

% \texttt{make clean}

% \item To remove the intermediate files, but leave the PostScript
% and DVI/PDF files intact, use the command:

% \texttt{make neat}
% \end{itemize}

% As you add or remove figures, chapters, or appendices to your thesis,
% make sure you keep the \texttt{Makefile} upto date, too (see the
% \texttt{FIGURES} and \texttt{FILES} macros in the \texttt{Makefile}).

% % \section{Changing Fonts}

% Change fonts: {\Large Large},
% \verb+verbatim ~@#$%^&*(){}[]+,
% \textsc{Small Caps},
% \textsl{slanted text},
% \emph{emphasized text},
% \texttt{typewriter text}.

% % \section{Accents and Ligatures}

% Some accents:
% \'{e}
% \`{e}
% \^{o}
% \"{u}
% \c{c}
% \"{\i}
% \'{\i}
% \~{n}
% \={a}
% \v{a}
% \u{a}

% \noindent Some ligatures:
% fl{\ae}ffi

\chapter{Figures and Tables}
\label{chap:figtab}

\section{Figures}

We can include encapsulated PostScript\texttrademark\ figures
(\texttt{.eps}) in the document and refer to it using a label.
For example, MUN's logo can be seen in Figure~\ref{fig:MUN_Logo_Pantone}.
\munepsfig{MUN_Logo_Pantone}{This is MUN's logo}

Figure~\ref{fig:enrollment} shows a chart of MUN's Fall
enrollment from 2005 -- 2009.\munfootnote{From \emph{Memorial
University of Newfoundland --- Fact Book 2009}.}
\munepsfig[scale=0.50]{enrollment}{MUN Fall Enrollment 2005 -- 2009}
The figure was created using the \textsf{Calc} spreadsheet application of
the office suite \textsf{OpenOffice.org}.\munfootnote{This office suite
can be downloaded at no cost from \texttt{http://openoffice.org/}. Unlike
other commercial office suites, \textsf{OpenOffice.org} may be legally
shared with colleagues and fellow students.  There are versions for
Linux, Microsoft Windows, Mac~OS~X and Solaris.  Also, unlike commercial
offerings, \textsf{OpenOffice.org} does not require activation using
registration keys.}  This figure was reduced by 50\%.

For larger figures, we can use landscape mode to rotate the page
and display the figure using the \verb+\munlepsfig+ command, as shown
in Figure~\ref{fig:enrollment-landscape}.  The figure will be the
only thing on the page when typeset in landscape mode. 
(The figure is reduced to 85\% of its original size.)
\munlepsfig[scale=0.85]{enrollment-landscape}
	{MUN Fall Enrollment 2005 -- 2009 (landscape)}

Alternatively, if we just want to rotate the figure, but not 
the entire page, we can specify an \texttt{angle} attribute
in the default argument of the \verb+\munepsfig+ command.
The result is shown in Figure~\ref{fig:enrollment-rotate}.
If the figure is too large or if there isn't sufficient
text, then the figure may appear on its own page.
\munepsfig[scale=0.30,angle=90]{enrollment-rotate}
	{MUN Fall Enrollment 2005 -- 2009 (rotated)}

Note that all three of the enrollment figures are basically
the same file, but with different names --- on Linux, they are
symbolic links to the same file.  The filenames have to be different
because the reference labels need to be unique.

Figure~\ref{fig:db-deadlock} shows a Petri net created using the
\texttt{xfig} program (\texttt{http://www.xfig.org/}) which has
very good support for \LaTeX.  This figure has been
reduced to 40\% of its original size.
\munepsfig[scale=0.40]{db-deadlock}{A deadlocked Petri net}

We can also create figures of text (such as short code snippets)
using the \verb+\muntxtfig+ command, as show in Figure~\ref{fig:code}.
\begin{muntxtfig}[1.0]{code}{Hello World}{0.5\textwidth}
\begin{verbatim}
#include <stdio.h>

int main(int argc, char **argv)
{
  printf("Hello world!\n");
  exit(0);
}
\end{verbatim}
\end{muntxtfig}

\section{Tables}

We can also create tables, as seen by Table~\ref{tab:pop}.  Note that,
as required by SGS guidelines, the caption for a table appears above the
table whereas figure captions appear below the figures.  Tables and
figures can ``float'' --- they may not appear on the page on which they
are mentioned.  \LaTeX{} tries to handle figure and table placement
intelligently, but if if you have a lot of them without a reasonable
amount of surrounding textual content, the figures and tables can
accumulate towards the end of the chapter.  Generally speaking, if
there is sufficient text explaining the tables and figures or if the
tables/figures are relatively small, this may not be a problem.  However,
if you have a lot of tables or figures, it may be a good idea to put
them in an appendix and refer to them as the need arises.

\begin{muntab}{c||c|c|c||c|c|c|}{pop}{Fall Semester Enrollment}
\hline
	& \multicolumn{3}{c||}{Undergraduate}
	& \multicolumn{3}{c|}{Graduate} \\
\cline{2-7}
     & F/T & P/T & Total & F/T & P/T & Total \\
\cline{2-7}
2004 & 13,191 & 2,223 & 15,414 & 1,308 & 879 & 2,187 \\
2005 & 13,184 & 2,143 & 15,327 & 1,375 & 920 & 2,295 \\
2006 & 12,809 & 2,224 & 15,033 & 1,373 & 899 & 2,272 \\
2007 & 12,634 & 2,155 & 14,789 & 1,403 & 899 & 2,302 \\
2008 & 12,269 & 2,208 & 14,477 & 1,410 &1,005& 2,415 \\
2009 & 12,382 & 2,323 & 14,705 & 1,567 &1,106& 2,673 \\
\hline
\end{muntab}

Table~\ref{tab:degrees} shows a different table in landscape
mode.\munfootnote{This data was also taken from the \emph{Memorial
University of Newfoundland --- Fact Book 2009}.} This is useful if your
table is too wide for the page.  Tables are double-spaced by default.
To single-space a table, change the \verb+\baselinestretch+ before
beginning the table environment.  Remember to restore it after the
environment has ended.

\renewcommand{\baselinestretch}{1.0}\normalsize
\begin{munltab}{lrrrrrrrrrrrr}
	{degrees}
	{Masters Degrees Conferred by Convocation Session --- 1950 to 2009}
\cline{2-13}
				&
\multicolumn{2}{|c|}{2009}	&
\multicolumn{2}{c|}{2008}	& 
\multicolumn{2}{c|}{2007}	& 
\multicolumn{2}{c|}{2006}	& 
\multicolumn{2}{c|}{2006}	& 
\multicolumn{1}{c|}{1950--2004}	& 
\multicolumn{1}{c|}{Total}	\\
\cline{2-13}
	  &
May & Oct &
May & Oct &
May & Oct &
May & Oct &
May & Oct & &  \\
Degrees \\
\hline
Master of Applied Science		&  14 &   2 &  15 &   8 &  28 &   1 &  21 &   3 &   3 &   1 &    98 &   194 \\
Master of Applied Social Psychology     &   1 &   5 &   2 &   5 &   1 &   4 &   0 &   4 &   0 &   4 &    28 &    54 \\
Master of Applied Statistics            &   0 &   0 &   3 &   1 &   0 &   0 &   1 &   0 &   0 &   0 &    19 &    24 \\
Master of Arts                          &  37 &  49 &  26 &  43 &  14 &  42 &  14 &  56 &  13 &  44 &   994 & 1,332 \\
Master of Business Administration       &  14 &  16 &  23 &   6 &  33 &  12 &  33 &  11 &  33 &   8 &   818 & 1,007 \\
Master of Education                     & 107 &  87 & 120 &  55 & 147 &  74 & 108 &  76 & 113 &  75 & 2,603 & 3,565 \\
Master of Employment Relations          &   8 &   9 &   5 &   7 &   7 &  14 &   4 &   9 &   3 &   5 &    12 &    83 \\
Master of Engineering                   &  20 &  19 &  20 &  10 &  16 &  10 &  15 &  13 &   4 &  19 &   440 &   586 \\
Master of Environmental Science         &   3 &   3 &   3 &   1 &   0 &   1 &   7 &   1 &   3 &   1 &    66 &    89 \\
Master of Marine Studies                &   2 &   0 &   0 &   1 &   0 &   2 &   2 &   2 &   1 &   2 &    26 &    38 \\
Master of Music                         &   4 &   1 &   5 &   0 &   3 &   0 &   3 &   0 &   3 &   0 &     7 &    26 \\
Master of Nursing                       &   7 &   8 &  10 &   4 &  17 &   4 &  23 &   7 &   6 &   1 &   116 &   203 \\
Master of Oil and Gas Studies           &   0 &   0 &   2 &   0 &   0 &   0 &   0 &   2 &   4 &   0 &     0 &     8 \\
Master of Philosophy                    &   5 &   4 &   2 &   1 &   5 &   2 &   5 &   3 &   2 &   0 &   112 &   141 \\
Master of Physical Education            &   0 &   2 &   3 &   0 &   5 &   4 &   3 &   0 &   4 &   4 &    84 &   109 \\
Master of Public Health                 &   0 &   8 &   0 &   0 &   0 &   0 &   0 &   0 &   0 &   0 &     0 &     8 \\
Master of Science                       &  40 &  32 &  41 &  19 &  29 &  25 &  35 &  29 &  32 &  23 & 1,653 & 1,958 \\
Master of Science (Kinesiology)         &   1 &   0 &   4 &   2 &   1 &   2 &   2 &   6 &   4 &   3 &     0 &    25 \\
Master of Science (Medicine)            &  18 &   7 &  11 &   8 &  10 &   5 &   9 &   9 &   8 &   4 &     0 &    89 \\
Master of Science (Pharmacy)            &   0 &   0 &   1 &   1 &   0 &   0 &   0 &   0 &   1 &   0 &    16 &    19 \\
Master of Social Work                   &   4 &  11 &   4 &   5 &   4 &   9 &   9 &   5 &   4 &  10 &   257 &   322 \\
Master of Women's Studies               &   2 &   0 &   2 &   0 &   1 &   1 &   2 &   3 &   2 &   0 &    20 &    33 \\
\hline
\textbf{Total Masters}                  & 287 & 263 & 302 & 177 & 321 & 212 & 296 & 239 & 243 & 204 & 7,369 & 9,913 \\
\end{munltab}
\renewcommand{\baselinestretch}{\spacing}\normalsize

\chapter{Dealing with Errors}
\label{chap:errors}

\LaTeX{} can produce cryptic error messages at times.
However, with some experience, it is usually not too
difficult to determine what the problem is and how to fix it.

As mentioned earlier, appropriate search terms in Google
may help you fix these error messages.


\chapter{Lorem Ipsum}
\label{chap:ch4_abbr}
Now, for your reading pleasure, some \textsl{Lorem ipsum}, courtesy
of:
\begin{center}
\texttt{<http://www.lipsum.com/>}
\end{center}
This gives a good view of the margins --- note that the left margin
is a bit wider than the right margin to accommodate binding.

Lorem ipsum dolor sit amet, consectetur adipiscing elit. Etiam odio elit,
viverra eu tempor non, pulvinar ac nisi. Pellentesque habitant morbi
tristique senectus et netus et malesuada fames ac turpis egestas. Sed
adipiscing, dui quis viverra facilisis, quam libero adipiscing justo,
vitae dictum libero mauris ac magna. Aenean sem ligula, vulputate at
vestibulum eu, pellentesque in justo. Sed et eros mauris, sed placerat
nulla. Maecenas nulla velit, facilisis et rutrum nec, volutpat id
lorem. Duis vestibulum odio velit, id elementum tortor. Sed pellentesque
leo ac nibh iaculis at fermentum orci lobortis. Suspendisse arcu magna,
porta nec pretium non, feugiat vitae orci. Vivamus at enim arcu,
at sagittis nisl. Vestibulum at mi enim, vel malesuada justo. Class
aptent taciti sociosqu ad litora torquent per conubia nostra, per
inceptos himenaeos. Nullam sed nunc at enim posuere sagittis. Vivamus
augue turpis, mattis a blandit non, sollicitudin non nisl. Integer
vestibulum, est vitae cursus adipiscing, elit libero pretium leo,
in scelerisque augue felis volutpat nisl. Donec commodo posuere arcu,
eget feugiat dui ornare nec. Nullam eros mi, condimentum ac ultricies ac,
euismod lobortis nibh. Cras ac ligula pharetra risus elementum pharetra
vel in quam. Fusce ac augue vulputate nibh imperdiet convallis sit amet
et quam. Integer porttitor dictum fermentum.

Nullam id ante arcu. Nulla facilisi. Vestibulum sodales, mi sodales
ultricies pulvinar, orci leo dictum diam, quis imperdiet turpis lacus
ut sem. Nulla rutrum odio sit amet elit aliquam blandit gravida nunc
placerat. Aenean et neque ut leo condimentum vehicula. Fusce quis orci
vitae enim dapibus tincidunt in vel ipsum. Phasellus auctor neque ac eros
egestas sit amet ultricies erat vestibulum. Ut erat ligula, pharetra
vel hendrerit vitae, mattis ac turpis. Ut malesuada diam vitae lacus
vestibulum a tempus nisl posuere. Ut nisi sem, dictum eu laoreet sed,
commodo eget enim. Morbi vel lacus neque, tempus fringilla tellus. Nunc
id egestas felis. Nullam eu mollis neque. Ut non mauris malesuada
eros sagittis congue. Cras vitae felis ut nisl mollis semper ut quis
risus. Sed eu arcu urna, et commodo sapien. Donec vestibulum, libero
sit amet ultrices blandit, erat lorem volutpat lectus, sed feugiat leo
elit in orci. Aliquam vitae leo tellus, placerat pulvinar massa. Nulla
at sapien hendrerit diam varius vehicula.

Curabitur et orci nulla. Phasellus euismod, massa non hendrerit dictum,
dolor enim imperdiet sapien, vitae commodo lorem tellus eu quam. Duis
egestas felis velit. Sed in orci nec nulla rutrum posuere. Suspendisse
potenti. Nunc vel quam nisi. In at molestie libero. Aenean hendrerit
vestibulum orci, ut hendrerit nulla volutpat lacinia. Vestibulum sit amet
sapien vitae lectus gravida vehicula. Suspendisse ac purus sit amet est
congue auctor.

Morbi pellentesque, quam vel mattis molestie, augue purus vestibulum
lorem, nec consequat enim eros eu augue. In odio dolor, scelerisque
a lobortis porttitor, commodo ut lacus. Maecenas sit amet diam
nec tellus accumsan bibendum. Praesent in turpis velit, malesuada
commodo sapien. Nunc ornare urna enim. Sed at diam non metus porttitor
suscipit. Aliquam erat volutpat. Duis aliquet magna in mauris semper
placerat. Ut eget quam orci. Ut egestas, dolor at dapibus accumsan, leo
nibh egestas urna, ac consectetur dui odio quis eros. Nam libero dolor,
lacinia eget imperdiet non, malesuada vehicula diam. Etiam id ipsum eget
turpis consectetur tristique id at ante. Vivamus blandit nunc eu nisl
varius sed accumsan odio molestie.


\include{chap5}
\chapter{Conclusions}
\label{chap:conclusions}
That's all folks!


\documentclass[a4paper,14pt]{extreport}
\usepackage[utf8]{inputenc}
\usepackage[english,russian]{babel}
% \usepackage[dvips]{graphicx}
\usepackage[pdftex]{graphicx}
\usepackage{geometry}
\geometry{left=25mm}
\geometry{right=20mm}
\geometry{top=20mm}
\geometry{bottom=20mm}
\graphicspath{{image/}}
\usepackage[usenames]{color}
\usepackage{colortbl}
\usepackage{multirow}
\usepackage{amssymb,amsfonts,amsmath,mathtext,cite,enumerate,float}
\usepackage{setspace}
\usepackage{indentfirst}
\onehalfspacing


\begin{document}
\begin{titlepage}
\begin{table}[]
    \centering
    \begin{tabular}{rcl}
    Автономная некоммерческая &
    \multirow{4}{*}{\includegraphics[width=40mm]{image/logo.eps}}
          & Autonomous noncommercial \\
    организация высшего  & & organization of higher \\
    образования & & education \\
    «Университет Иннополис»  &
     & «Innopolis University» \\
    \hline
    \hline
    \end{tabular}
    \label{tab:my_label}
\end{table}
\vline
\vspace{20mm}

\begin{center}
\textbf{АННОТАЦИЯ \\ НА ВЫПУСКНУЮ КВАЛИФИКАЦИОННУЮ РАБОТУ  \\
ПО НАПРАВЛЕНИЮ ПОДГОТОВКИ \\ 09.03.01 --- «ИНФОРМАТИКА И ВЫЧИСЛИТЕЛЬНАЯ ТЕХНИКА»}
\end{center}
\vspace{20mm}


    \begin{tabular}{ll
|>{\columncolor[gray]{.9}}l|}
\cline{3-3}
\textbf{Тема:} &
     &
    \makebox[133mm][l]{Альтернативы статической типизации для Clojure}    \\
    &&\\
    && \\
    &&  \\
\cline{3-3}
    \end{tabular}
\vspace{5mm}


    \begin{tabular}{ll
|>{\columncolor[gray]{.9}}l|l
|>{\columncolor[gray]{.9}}l|}
\cline{3-3} \cline{5-5}
Выполнил &
     &
    \makebox[77mm][l]{Тропин Андрей Геннадьевич}   &
    &    \\
    &&&&
    \makebox[39.5mm]{\textcolor[gray]{.7}{подпись}} \\
    &&&& \\
    \cline{3-3} \cline{5-5}
    \end{tabular}
\vspace{5mm}

\vspace{\fill}

\begin{center}
Иннополис, 2017
\end{center}
\end{titlepage}


\newpage
% \noindent {\large \textbf{Содержание}} \\

\chapter*{Содержание}
\documentclass[dvips,letterpaper,12pt]{report}
\usepackage{thesis}

\begin{document}

\pagenumbering{roman}

% Fill in the title, author, degree name, department, and month/year.
% Upon completion, this should look like the following:
%\thesistitle
%	{Complicated and Important-Sounding Thesis Title}
%	{John P. Doe}
%	{Master of Science}
%	{Department of Computer Science}
%	{May 2009}
% The \thesistitle definition is in thesis.sty.  Other customizations
% can be made there.
\thesistitle
	{Thesis: \\
	 Alternatives of static type system and it's application for Clojure
programming language.\\
	 Submission to the School of Graduate Studies  \\}
	{\emph{Andrew Tropin}}
	{Bachelor of \emph{Science}}
	{Department of \emph{Computer Science}}
	{\emph{Month Year}}

\include{abstract}
\include{ack}
\include{contents}
\include{tables}
\include{figures}

\pagenumbering{arabic}
\include{chap1}
\include{chap2}
\include{chap3}
\include{chap4}
\include{chap5}
\include{chap6}

\include{bib}

% If you have no appendices, remove the following two lines.
% If you have more appdences, add them as necessary.
\appendix
\include{apdxa}

\end{document}


% \newpage
% \noindent {\large \textbf{Введение}}
% \newline
\chapter*{Введение}

Существует два основных вида систем типизации, применяемых в современных языках
программирования: статическая и динамическая. Сторонники первого подхода
говорят, что преимущества статической проверки типов включает в себя: более
раннее обнаружение ошибок (например, предотвращает сложение переменных
строкового и целочисленного типа), дополнительную документацию и метаданные для
прогрессивного автодополнения фрагментов кода средствами интегрированной среды
разработки, а также возможность предоставлять компилятору вспомогательную
информацию, позволяющую производить оптимизации во время компиляции (например,
замена виртуальных вызовов прямыми вызовами функций).

С другой стороны, есть и сторонники динамической проверки типов, которые
считают, что языки с таким подходом имеют следующие преимущества: интерактивный
способ разработки (например, разработка на основе техник с использование REPL +
TDD, где тесты могут быть запущены без перекомпиляции всего проекта), быстрое
прототипирование (например, решения касательно типов могут быть отложены, язык
гораздо более выразителен и больше подходит для разработки систем с быстро
изменяющимися или неизвестными требования).

Статическая проверка типов - хороший инструмент, но нужно понимать, что он не
гарантирует, что программы будут безошибочными и правильными, поскольку он
обеспечивает всего лишь абстракцию во время компиляции над поведением запущенной
программной системы, это значит, что некоторые ошибки могут быть обнаружены, но
выполнение программы всё же может пойти не так. Другое, более обширное
объяснение, представленное Дэвидом Маклвером в его работе [11] более подробно
обозначает причины по которым невозможно предотвратить все ошибки во время
компиляции. Кроме того, статическая типизации заставляет разрбаотчика принимать
решения раньше, замедляет процесс компиляции и делает интерактивную разработку
практически невозможной.

В настоящее время многие люди / команды / компании заинтересованы в создании
систем оперирующих с большими объёмами данных, просто взгляните на текущие
тенденции: BigData, машинное обучение становятся все более популярными. Причина
ясна: огромные количество данных позволяет извлечь полезные знания, но для этого
нужны подходящие инструменты. Вероятно, динамичность является одним из наиболее
важных свойств языка для такого рода программных систем, поскольку большинство
данных не полностью структурировано, согласно исследовательскому отчету по
проекту Университета Беркли [10] около $95\%$. В случаях когда структура
известна заранее по прошествии нескольких шагов, люди или алгоритмы генерируют
запросы, основанные на информации и данных времени выполнения, также приобретают
динамический характер.

Динамическая проверка типов - хороший инструмент для интенсивно использующих
данные или других типов динамических систем, но важно понимать недостатки такого
подхода. Как правило приозводить рефакторинг, разрабатывать и поддерживать
системы, использующие языки с динамической типизацией сложнее, так как любое
даже небольшое изменение может в перспективе привести к краху системы, и никто
не узнает об этом, пока ошибка не возникнет позже во время выполнения. Есть
несколько техник, которые помогают справиться с проблемами такого рода, например
разработка через тестирование [1] или разработка на основе контрактов [14], но
они не решают всех проблем. Иногда удобно декларировать форму данных с помощью
спецификаций, чтобы впоследствии проверять, соответствуют ли текущие данные
формату или создавать образцы данных из таких спецификаций.

Информацию о том, когда и какой вид системы типизации лучше использовать, можно
найти в статье [13], авторы говорят, что по возможности нужно использовать
статическую типизацию и только, если необходимо - динамическую, но что, если нет
возможности выбрать систему проверки типов и необходимо использовать динамически
типизированный язык? Могут ли быть преимущества статической проверки типов
получены в таком окружении?

\chapter*{Основная часть}
% \chapter*{Заключени}



\documentclass[dvips,letterpaper,12pt]{report}
\usepackage{thesis}

\begin{document}

\pagenumbering{roman}

% Fill in the title, author, degree name, department, and month/year.
% Upon completion, this should look like the following:
%\thesistitle
%	{Complicated and Important-Sounding Thesis Title}
%	{John P. Doe}
%	{Master of Science}
%	{Department of Computer Science}
%	{May 2009}
% The \thesistitle definition is in thesis.sty.  Other customizations
% can be made there.
\thesistitle
	{Thesis: \\
	 Alternatives of static type system and it's application for Clojure
programming language.\\
	 Submission to the School of Graduate Studies  \\}
	{\emph{Andrew Tropin}}
	{Bachelor of \emph{Science}}
	{Department of \emph{Computer Science}}
	{\emph{Month Year}}

\include{abstract}
\include{ack}
\include{contents}
\include{tables}
\include{figures}

\pagenumbering{arabic}
\include{chap1}
\include{chap2}
\include{chap3}
\include{chap4}
\include{chap5}
\include{chap6}

\include{bib}

% If you have no appendices, remove the following two lines.
% If you have more appdences, add them as necessary.
\appendix
\include{apdxa}

\end{document}

\end{document}


% If you have no appendices, remove the following two lines.
% If you have more appdences, add them as necessary.
\appendix
\chapter{Appendix title}
\label{apdx:somelabel}
This is Appendix~\ref{apdx:somelabel}.

You can have additional appendices too
(\emph{e.g.}, \texttt{apdxb.tex}, \texttt{apdxc.tex}, \emph{etc.}).
If you don't need any appendices, delete the appendix
related lines from \texttt{thesis.tex} and the file names
from \texttt{Makefile}.


\end{document}

\end{document}
