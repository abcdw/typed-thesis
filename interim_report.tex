%%% TeX-command-extra-options: "-shell-escape"
\documentclass[12pt,a4paper]{report}

%\usepackage[tmargin=2cm, bmargin=2.5cm]{geometry}
\usepackage{titlesec}
\usepackage{enumitem}
\usepackage{bookmark}
\usepackage{hyperref}
\usepackage{sectsty}
\usepackage{listings}
\ifx\pdfoutput\undefined
\usepackage{graphicx}
\else
\usepackage[pdftex]{graphicx}
\fi
\usepackage{indentfirst}
\usepackage{hyperref}
\usepackage{minted}

\sectionfont{\centering}

\begin{document}

\begin{titlepage}
	\centering
	% \includegraphics[width=0.15\textwidth]{example-image-1x1}\par\vspace{1cm}
	{\scshape\LARGE Innopolis University \par}
	\vspace{1cm}
	{\scshape\Large Final thesis interim progress report \par}
	\vspace{1.5cm}
	{\huge\bfseries
    Alternatives of static type system and it's application for Clojure programming language
    \par}
	\vspace{2cm}
	{\Large\itshape Andrew Tropin\\
    andrewtropin@gmail.com
    \par}
	\vfill
	supervised by\par
	Manuel Mazzara, Alexander Chichigin

	\vfill

% Bottom of the page
	{\large \today\par}
\end{titlepage}

\section*{Goals of the Project}

The main goal of the project is to create tool, which partially compensates lack
of static type system for clojure programming language.\\

More precisely it covers:
\begin{itemize}
\item Documentation, provided by type annotations
\item Ability to catch errors earlier and closer to problem location
\item Refactoring with greater confidence
\item Better error messages
\end{itemize}

There are also some issues, which static type system helps to solve, but they are not
addressed in this work:
\begin{itemize}
\item Faster resulting programs
  (see \href{https://www.youtube.com/watch?v=yGko70hIEwk&t=295s}{related work})
\end{itemize}

Most of following ideas can be applied for all functional languages with dynamic type system.\\

First of all it's necessary to understand, that code written in functional
language with persistent data structures are easy to test. It's especially
necessary to write tests for dynamically typed language, because it's very
hard to trace errors introduced earlier, it will be found only in runtime
and can be very far from place where exception raised. \\

Tests are good, but someone have to write them, also code in dynamically
typed language have lack of information about function and parameters
types.\\

One of the solutions is to write optional type annotations,
which can be used for generating project documentation/docstrings
and tests.

\section*{System Specification}

There are already many projects using custom annotations for their needs. For
example \textbf{prismatic.schema}, \textbf{herbert}, \textbf{core.typed} and so
on. Main problem is that tools are not supported by language developers, some
of them force to change code drastically and sometimes doesn't follow clojure
way. Solution is to use annotations which supported by core library of the
language, it's guarantied to be in any distribution of compiler and follow
language coans.

In next major release of \textbf{Clojure}(1.9) will be introduced tool called
\textbf{clojure.spec} as a part of language core. It is domain specific
language, which allows to write composition of predicates to describe shape and
type of data. Another features of specs is data coersion, detailed explanation
of predicate violation and generating of sample data conforming a particular
spec. Some examples are illustrated below. \\

\begin{minted}{clojure}
(def nillable-string?
  (s/or :name string? :nil nil?))

(s/def ::email
  (s/and
   string?
   #(re-matches #"^[a-zA-Z0-9._%+-]+@[a-zA-Z0-9.-]+\.[a-zA-Z]{2,63}" %)))

(s/def ::new-user
  (s/merge (s/map-of #{:password :full_name :email :phone} nillable-string?)
           (s/keys :req-un [::email] :opt-un [::password ::full_name ::phone])))
\end{minted}

Tool is based on \textbf{clojure.spec} and allows to generate docstrings for current
repl session, using existing annotations. \\

Another feature of \textbf{clojure.spec} is ability to generate sample data from
specs:

\begin{minted}{clojure}
(s/def ::kws (s/with-gen (s/and keyword? #(= (namespace %) "my.domain"))
               #(s/gen #{:my.domain/name :my.domain/occupation :my.domain/id})))
(s/valid? ::kws :my.domain/name)  ;; true
(gen/sample (s/gen ::kws))
;;=> (:my.domain/occupation :my.domain/occupation :my.domain/name  ...)

(s/def ::roll (s/int-in 0 11))
(gen/sample (s/gen ::roll))
;;=> (1 0 0 3 1 7 10 1 5 0)
\end{minted}

Using this feature tool can generate tests for annotated functions. If function
failing test and result doesn't conform provided spec, it's easy to get
explanation such that:

\begin{minted}{clojure}
(s/explain-data ::name-or-id :foo)
;;=> #:clojure.spec{
;;     :problems ({:path [:name],
;;                 :pred string?,
;;                 :val :foo,
;;                 :via [:spec.examples.guide/name-or-id],
;;                 :in []}
;;                {:path [:id],
;;                 :pred int?,
;;                 :val :foo,
;;                 :via [:spec.examples.guide/name-or-id],
;;                 :in []})}
\end{minted}

Docstring generation and automatic test generation is two main goals of the tool.

\section*{Background theory}

Information sources for this thesis can be split in three categories.\\

\begin{itemize}
\item First is thoughts about advantages and disadvantages of static type system
and experience of different companies. \cite{coldwar} - \cite{statictypingadv}
\item Second is about existing solutions, its implementation and problems.
  \cite{contracts} - \cite{typedscheme}
\item Last one is about clojure.spec. \cite{customdefn} - \cite{baremetal}
\end{itemize}

Most valuable thoughts of papers and articles below are conducted and presented in first two
sections of this report.

\section*{Task specification and project schedule}
This project uses agile results methodology. \\

Project milestones, first part of each milestone is developing, second is
writing documentation and reports: \\
\begin{itemize}
\item Include specs in doc strings(Dec-Jan)
\item Generate samples for tests using specs(Feb)
\item Provide simplest sample data for failed test(Mar-Apr)
\end{itemize}

More information about current tasks and milestones can be found on \href{https://github.com/abcdw/typed-thesis/projects/1}{https://github.com/abcdw/typed-thesis}

% \section*{Review of tasks}


\section*{Interim results and short-term plans}

Project ideas inspired by experience of developing REST API service(urbest), using
test/repl driven development and researches about language type systems
\cite{coldwar} - \cite{statictypingadv}. After few months of development
it became obvious that \textbf{clojure} projects have to have good test
coverage, but many test cases are simple and can be generated from description
of the shape of data, also it will be very helpful to have additional
information about types of arguments in docstring. That's why few specifications
was added to urbest project to help understand some parts of codebase and to
make it easier to validate passed parameters and add post- and pre-conditions
and write test.\\

First of all was created tool, which helps to update docstrings for
current repl session, using existing annotations. At the moment tool, which add
information about parameters and result type to
docstring is just few functions in one of the namespaces of the urbest project.
Nearest plans is to make a separate library and move all functionality connected
with docstring generation to it.\\





% \section*{Short term plans}


\begin{thebibliography}{9}
\bibitem{coldwar}
  Static Typing Where Possible, Dynamic Typing When Needed: The End of the Cold
  War Between Programming Languages.\\
  \textbf{https://www.ics.uci.edu/\textasciitilde{}lopes/teaching/inf212W12/readings/rdl04meijer.pdf}

\bibitem{cicrlecicoretyped}
  Why we're no longer using core typed.\\
  \textbf{https://circleci.com/blog/why-were-no-longer-using-core-typed/}
\bibitem{brokensoftware}
  Static typing will not save us from broken software.\\
  \textbf{http://www.drmaciver.com/2016/10/static-typing-will-not-save-us-from-broken-software/}
\bibitem{statictypingadv}
  The advantages of static typing, simply stated.\\
  \textbf{https://pchiusano.github.io/2016-09-15/static-vs-dynamic.html}



\bibitem{contracts}
  On Contract Satisfaction in a Higher-Order World.\\
  \textbf{http://www.ccs.neu.edu/racket/pubs/df-toplas11.pdf}
\bibitem{typedx}
  Typed X just doesn't pan out?\\
  \textbf{http://lambda-the-ultimate.org/node/5297}
\bibitem{gradualtyping}
  Is Sound Gradual Typing Dead?\\
  \textbf{http://www.ccs.neu.edu/racket/pubs/popl16-tfgnvf.pdf}
\bibitem{typedscheme}
  The Design and Implementation of Typed Scheme.\\
  \textbf{http://www.ccs.neu.edu/racket/pubs/popl08-thf.pdf}
% - [ ] [A Comparative Study of Programming Languages in Rosetta Code](http://arxiv.org/pdf/1409.0252.pdf)



\bibitem{customdefn}
  Custom defn macro with clojure.spec.\\
  \textbf{http://blog.klipse.tech//clojure/2016/10/10/defn-args.html}
\bibitem{autoannotations}
  Automatic annotations(typed+spec).\\
  \textbf{http://frenchy64.github.io/2016/08/07/automatic-annotations.html}
\bibitem{aboutspec}
  About clojure.spec. \\
  \textbf{http://clojure.org/about/spec}
\bibitem{baremetal}
  Bare Metal clojure.\\
  \textbf{https://www.youtube.com/watch?v=yGko70hIEwk}

\end{thebibliography}

\end{document}