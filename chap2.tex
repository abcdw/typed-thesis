\chapter{Background}
\label{chap:background}

\section{Interactive programming}
Explains tools and advantages of such development style, helps to understand why
particular benefits was chosen for implementing.

\section{Benefits and alternatives}
Benefits of static type checking and their alternatives, which implemented in thesis


\section{Optional annotations}
What can be done prismatic.schema, clojure.spec

\section{Gradual typing}
\subsection{TypedClojure}
TypedClojure, pros and cons
\subsection{Spectrum}
spectrum

\section{Draft figures}

We can include encapsulated PostScript\texttrademark\ figures
(\texttt{.eps}) in the document and refer to it using a label.
For example, MUN's logo can be seen in Figure~\ref{fig:MUN_Logo_Pantone}.
\munepsfig{MUN_Logo_Pantone}{This is MUN's logo}

Figure~\ref{fig:enrollment} shows a chart of MUN's Fall
enrollment from 2005 -- 2009.\munfootnote{From \emph{Memorial
University of Newfoundland --- Fact Book 2009}.}
\munepsfig[scale=0.50]{enrollment}{MUN Fall Enrollment 2005 -- 2009}
The figure was created using the \textsf{Calc} spreadsheet application of
the office suite \textsf{OpenOffice.org}.\munfootnote{This office suite
can be downloaded at no cost from \texttt{http://openoffice.org/}. Unlike
other commercial office suites, \textsf{OpenOffice.org} may be legally
shared with colleagues and fellow students.  There are versions for
Linux, Microsoft Windows, Mac~OS~X and Solaris.  Also, unlike commercial
offerings, \textsf{OpenOffice.org} does not require activation using
registration keys.}  This figure was reduced by 50\%.

For larger figures, we can use landscape mode to rotate the page
and display the figure using the \verb+\munlepsfig+ command, as shown
in Figure~\ref{fig:enrollment-landscape}.  The figure will be the
only thing on the page when typeset in landscape mode.
(The figure is reduced to 85\% of its original size.)
\munlepsfig[scale=0.85]{enrollment-landscape}
	{MUN Fall Enrollment 2005 -- 2009 (landscape)}

Alternatively, if we just want to rotate the figure, but not
the entire page, we can specify an \texttt{angle} attribute
in the default argument of the \verb+\munepsfig+ command.
The result is shown in Figure~\ref{fig:enrollment-rotate}.
If the figure is too large or if there isn't sufficient
text, then the figure may appear on its own page.
\munepsfig[scale=0.30,angle=90]{enrollment-rotate}
	{MUN Fall Enrollment 2005 -- 2009 (rotated)}

Note that all three of the enrollment figures are basically
the same file, but with different names --- on Linux, they are
symbolic links to the same file.  The filenames have to be different
because the reference labels need to be unique.

Figure~\ref{fig:db-deadlock} shows a Petri net created using the
\texttt{xfig} program (\texttt{http://www.xfig.org/}) which has
very good support for \LaTeX.  This figure has been
reduced to 40\% of its original size.
\munepsfig[scale=0.40]{db-deadlock}{A deadlocked Petri net}

We can also create figures of text (such as short code snippets)
using the \verb+\muntxtfig+ command, as show in Figure~\ref{fig:code}.
\begin{muntxtfig}[1.0]{code}{Hello World}{0.5\textwidth}
\begin{verbatim}
#include <stdio.h>

int main(int argc, char **argv)
{
  printf("Hello world!\n");
  exit(0);
}
\end{verbatim}
\end{muntxtfig}

\section{Draft Tables}

We can also create tables, as seen by Table~\ref{tab:pop}.  Note that,
as required by SGS guidelines, the caption for a table appears above the
table whereas figure captions appear below the figures.  Tables and
figures can ``float'' --- they may not appear on the page on which they
are mentioned.  \LaTeX{} tries to handle figure and table placement
intelligently, but if if you have a lot of them without a reasonable
amount of surrounding textual content, the figures and tables can
accumulate towards the end of the chapter.  Generally speaking, if
there is sufficient text explaining the tables and figures or if the
tables/figures are relatively small, this may not be a problem.  However,
if you have a lot of tables or figures, it may be a good idea to put
them in an appendix and refer to them as the need arises.

\begin{muntab}{c||c|c|c||c|c|c|}{pop}{Fall Semester Enrollment}
\hline
	& \multicolumn{3}{c||}{Undergraduate}
	& \multicolumn{3}{c|}{Graduate} \\
\cline{2-7}
     & F/T & P/T & Total & F/T & P/T & Total \\
\cline{2-7}
2004 & 13,191 & 2,223 & 15,414 & 1,308 & 879 & 2,187 \\
2005 & 13,184 & 2,143 & 15,327 & 1,375 & 920 & 2,295 \\
2006 & 12,809 & 2,224 & 15,033 & 1,373 & 899 & 2,272 \\
2007 & 12,634 & 2,155 & 14,789 & 1,403 & 899 & 2,302 \\
2008 & 12,269 & 2,208 & 14,477 & 1,410 &1,005& 2,415 \\
2009 & 12,382 & 2,323 & 14,705 & 1,567 &1,106& 2,673 \\
\hline
\end{muntab}

Table~\ref{tab:degrees} shows a different table in landscape
mode.\munfootnote{This data was also taken from the \emph{Memorial
University of Newfoundland --- Fact Book 2009}.} This is useful if your
table is too wide for the page.  Tables are double-spaced by default.
To single-space a table, change the \verb+\baselinestretch+ before
beginning the table environment.  Remember to restore it after the
environment has ended.

\renewcommand{\baselinestretch}{1.0}\normalsize
\begin{munltab}{lrrrrrrrrrrrr}
	{degrees}
	{Masters Degrees Conferred by Convocation Session --- 1950 to 2009}
\cline{2-13}
				&
\multicolumn{2}{|c|}{2009}	&
\multicolumn{2}{c|}{2008}	&
\multicolumn{2}{c|}{2007}	&
\multicolumn{2}{c|}{2006}	&
\multicolumn{2}{c|}{2006}	&
\multicolumn{1}{c|}{1950--2004}	&
\multicolumn{1}{c|}{Total}	\\
\cline{2-13}
	  &
May & Oct &
May & Oct &
May & Oct &
May & Oct &
May & Oct & &  \\
Degrees \\
\hline
Master of Applied Science		&  14 &   2 &  15 &   8 &  28 &   1 &  21 &   3 &   3 &   1 &    98 &   194 \\
Master of Applied Social Psychology     &   1 &   5 &   2 &   5 &   1 &   4 &   0 &   4 &   0 &   4 &    28 &    54 \\
Master of Applied Statistics            &   0 &   0 &   3 &   1 &   0 &   0 &   1 &   0 &   0 &   0 &    19 &    24 \\
Master of Arts                          &  37 &  49 &  26 &  43 &  14 &  42 &  14 &  56 &  13 &  44 &   994 & 1,332 \\
Master of Business Administration       &  14 &  16 &  23 &   6 &  33 &  12 &  33 &  11 &  33 &   8 &   818 & 1,007 \\
Master of Education                     & 107 &  87 & 120 &  55 & 147 &  74 & 108 &  76 & 113 &  75 & 2,603 & 3,565 \\
Master of Employment Relations          &   8 &   9 &   5 &   7 &   7 &  14 &   4 &   9 &   3 &   5 &    12 &    83 \\
Master of Engineering                   &  20 &  19 &  20 &  10 &  16 &  10 &  15 &  13 &   4 &  19 &   440 &   586 \\
Master of Environmental Science         &   3 &   3 &   3 &   1 &   0 &   1 &   7 &   1 &   3 &   1 &    66 &    89 \\
Master of Marine Studies                &   2 &   0 &   0 &   1 &   0 &   2 &   2 &   2 &   1 &   2 &    26 &    38 \\
Master of Music                         &   4 &   1 &   5 &   0 &   3 &   0 &   3 &   0 &   3 &   0 &     7 &    26 \\
Master of Nursing                       &   7 &   8 &  10 &   4 &  17 &   4 &  23 &   7 &   6 &   1 &   116 &   203 \\
Master of Oil and Gas Studies           &   0 &   0 &   2 &   0 &   0 &   0 &   0 &   2 &   4 &   0 &     0 &     8 \\
Master of Philosophy                    &   5 &   4 &   2 &   1 &   5 &   2 &   5 &   3 &   2 &   0 &   112 &   141 \\
Master of Physical Education            &   0 &   2 &   3 &   0 &   5 &   4 &   3 &   0 &   4 &   4 &    84 &   109 \\
Master of Public Health                 &   0 &   8 &   0 &   0 &   0 &   0 &   0 &   0 &   0 &   0 &     0 &     8 \\
Master of Science                       &  40 &  32 &  41 &  19 &  29 &  25 &  35 &  29 &  32 &  23 & 1,653 & 1,958 \\
Master of Science (Kinesiology)         &   1 &   0 &   4 &   2 &   1 &   2 &   2 &   6 &   4 &   3 &     0 &    25 \\
Master of Science (Medicine)            &  18 &   7 &  11 &   8 &  10 &   5 &   9 &   9 &   8 &   4 &     0 &    89 \\
Master of Science (Pharmacy)            &   0 &   0 &   1 &   1 &   0 &   0 &   0 &   0 &   1 &   0 &    16 &    19 \\
Master of Social Work                   &   4 &  11 &   4 &   5 &   4 &   9 &   9 &   5 &   4 &  10 &   257 &   322 \\
Master of Women's Studies               &   2 &   0 &   2 &   0 &   1 &   1 &   2 &   3 &   2 &   0 &    20 &    33 \\
\hline
\textbf{Total Masters}                  & 287 & 263 & 302 & 177 & 321 & 212 & 296 & 239 & 243 & 204 & 7,369 & 9,913 \\
\end{munltab}
\renewcommand{\baselinestretch}{\spacing}\normalsize
