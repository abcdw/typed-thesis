\chapter{Background}
\label{chap:background}

\section{Clojure}

Clojure is a dynamic language for the Java Virtual Machine, with a compelling
combination of features: it is elegant, clean, careful design lets you write
programs that get right to the essence of a problem, without a lot of clutter
and ceremony. Clojure is Lisp reloaded, it has the power inherent from Lisp, but
is not constrained by the history of Lisp. It is a functional language, where
data structures are immutable, and functions tend to be side-effect free. This
makes it easier to write correct programs, and to compose large programs from
smaller ones. Language designed for concurrency, rather than error-prone
locking, it provides software transactional memory. It has good Java
interopability functionality. Calling from Clojure to Java is direct, and goes
through no translation layer. It is fast. Wherever you need it, you can get
the exact same performance that you could get from hand-written Java code. Many
other languages offer some of these features, but the combination of them all
makes Clojure sparkle.

As for me first of all it is enjoyable, but if we talk about some boring
enterprise things: customers and stakeholders have substantial investments in
and are already comfortable with the performance, security and stability of
industry-standard platforms like the JVM. While Java developers may envy the
succinctness, flexibility and productivity of dynamic languages, they have
concerns about running on customer-approved infrastructure, access to their
existing code base and libraries, and performance.

Immutability and STM. In addition, they face ongoing problems dealing with
concurrency using native threads and locking. Clojure is an effort in pragmatic
dynamic language design in this context. It is created to be a general-purpose
language suitable in those areas where Java is suitable. It reflects the reality
that, for the concurrent programming future, pervasive, unmoderated mutation
simply has to go.

Clojure meets its goals by: embracing an industry-standard, open platform - the
JVM, modernizing a venerable language - Lisp, fostering functional programming
with immutable persistent data structures, and providing built-in concurrency
support via software transactional memory and asynchronous agents. The result is
robust, practical, and fast.

We can't handle power of machines with current languages. Java and OO stuff is
so complicated you can't refactor it and you can't fix it (Dave Thomas). It
means that it is easier to buy more cheap hardware and use some tools, which
maybe not so lightweight, but allows to create pretty simple architectures.

Another very important feature of the Clojure is a community. The community is
friendly, helpful and pretty smart. Most of the ideas implemented in other
languages have their own implementation in Clojure(Script). For example go
channels implemented as core.async library. React.js implemented at least in 4
ClojureScript “frameworks”: Reagent, om, om.next, Rum. Also, there are many
helpful resources and training courses already created.

Modern implementation of venerable lisp with advanced concurrency support on
mature enterprise ready platforms. It at least deserves attention. Finding
missing parts of such a great tool is a good contribution. More extensive
explanation about language and it's features can be found in
\cite{halloway2009programming}, \cite{fogus2011joy} and
\cite{hickey2008clojure}.


\section{Live programming}

There are two ways a do-it-yourselfer might replace an old lightswitch with a
dimmer: (1) first turn off the circuit breaker, or (2) wire it hot. Hot wiring
has two advantages: it’s probably faster, and in some cases it may be easier to
tell which wire is which by touching them to a light bulb or voltmeter. However,
hot wiring is dangerous.

In programming, working live need not be dangerous, and the opportunity it
offers for immediate feedback can be very valuable. Here are some of the
motivations for liveness in programming:

\begin{itemize}
\item Minimizing the latency between a programming action and seeing its effect
  on program execution
\item Allowing performances in which programmer actions control the dynamics of
  the audience experience in real time
\item Simplifying the “credit assignment problem” faced by a programmer when
  some programming actions induce a new runtime behavior (such as a bug)
\item Supporting learning (hence the early connections between liveness with
  visual programming and program visualization)
\end{itemize}

The traditional program development cycle involved the four separate phases:
edit, compile, link, run. Debugging sometimes altered the cycle by changing the
run mode to include setting breakpoints, single-stepping, etc. Changing a
program while it was being executed was rare outside of debugging sessions, and
the changes made during debugging were more often to data values than machine
code. Changes to the code were difficult to make, and when they were made, it
was generally while execution was suspended at a breakpoint. Live programming
was very much an exception to the norm.

In live programming, there is only one phase, at least in principle. The phase
involves the program constantly running, even as various editing events occur. A
system that supports live programming need not require that all programming
performed within the system be live. At times, live programming is unnecessary
and the execution of the program might be distracting, particularly when the
program is in an intermediate state between useful versions with meaningful
behavior.

In modern integrated development environments such as Emacs for Lisp or Eclipse
for Java programming, there are many features that work as the programmer codes,
to provide feedback to the programmer. These include syntax highlighting, code
completion suggestions, and indications of problems associated with various
locations in a source file. Facilities for editing running code also exist. The
Java virtual machine from version 1.4 has included a “hot-swap” feature that
enables the replacement of a class file by a new one while the overall program
is running. That permits IDEs to offer a code “push” feature to quickly compile
a new version of a class and/or object and insert it into the running JVM. When
programmers code live, their behavior may be different than without liveness,
and this can lead to new means to infer structures such as unit tests. About
future of live development Tanimoto and Steven L. talk in the
\cite{tanimoto2013perspective}.

Clojure is pretty good for live programming and rapid development as it uses
REPL (read eval print loop, like interactive command shell), has a good explicit
state management and it also suitable for TDD (test driven development) as it
has immutable and pure functional nature. Those workflows give huge productivity
boost \cite{madeyski2007impact}, \cite{tanimoto2013perspective}, that is why it
necessary to keep in mind this workflows when think about possible solutions.


\section{Static type checking}
\label{sec:statictypechecking}
It is necessary to clarify what advantages and disadvantages static type
checking has, first of all, to understand what benefits can be taken and
secondly what can be and should be avoided \cite{staticvsdynamic}. This also
important to not break existing workflows and keep feedback cycle very short. A
comparative study of programming languages \cite{nanz2015comparative} tells us
about what languages a most suitable for. From an engineering viewpoint, the
design of a programming language is the result of multiple tradeoffs that
achieve certain desirable properties, let's formulate what tradeoffs static type
checking has.

Clojure is a hosted language this means that it is not very strictly related to
concrete platform and can be compiled in different languages (javascript for
example) or vms bytecode (java virtual machine bytecode, common language runtime
bytecode). Cause of so much compilation/translation options are exists, it will
be very hard to talk about performance optimizations related to static typing,
that is why performance questions are not addressed in this work, but the impact
of optional type information on jit compilation of dynamically typed languages
are presented in \cite{chang2011impact}.

\subsection{Advantages}
First of all, a large class of errors are caught, earlier in the development
process, closer to the location where they are introduced, but it doesn't mean
that the code cannot be broken. \textbf{Example}: you have a map whose keys are
strings and whose values you expect (elsewhere in your program) to be functions
from Int -> Int, and you accidentally insert a function that returns a list of
integers. With static typing, this error would be caught by the typechecker and
pointed out to the programmer at the location the erroneous value is inserted.
In a dynamic language, the error could go unnoticed until much later, when that
value gets pulled out of the map and the function applied. As another example of
this sort of thing, many languages have a concept of null (the ‘billion dollar
mistake’). Even static languages like Java allow null to be used in place of any
type, which leads to a situations where unexpected null values pop up in your
program very far from the place where they were erroneously introduced. Such
problems mostly solved in programming languages like Kotlin.

The types guide development and can even write code for you, the programmer ends
up with less to specify. \textbf{Example}: Haskell has a
feature called typeclasses, which derives aspects of your program, purely based
on the types, which are often inferred. The resulting code can be noticeably
shorter than dynamic languages which cannot disambiguate programmer intent via
types. More advanced dependently typed languages routinely derive large amounts
of boring code with guidance from the programmer and the types specified. Conor
McBride likes to emphasize that types are not merely about preventing
errors, they are about declaring more of your intent to the machine so the
machine can do more work on your behalf.

One can refactor with greater confidence, since a large class of errors
introduced during refactoring will end up as type errors. Some techniques
discussed earlier (TDD, RDD) also gives same or similar benefits.
\textbf{Example}: you have a function that currently returns a single number,
and wish to modify it to return a list of numbers. In a static language,
updating the declared type signature and fixing any compile errors will catch
most, if not all places that need updating.

Static types can ease the mental burden of writing programs, by automatically
tracking information the programmer would otherwise have to track mentally in
some fashion. \textbf{Example}: you are writing a tricky function, and have two
values in scope, f and x. The f you pulled out of a dictionary, the x was the
result of calling foo(23). Is it safe to call f(x)? Types provide an automated,
precise answer to this question.

Types serve as documentation for yourself and other programmers and provide a
‘gradient’ that tells you what terms make sense to write. For someone trained,
the types give a sense of what is expressible using any API. Like puzzle pieces
with shapes that we can observe fit together, we can think of types as
specifying a grammar for programs that ‘make sense’. In dynamic languages (what
Bob Harper somewhat trollishly refers to as ‘singly-typed’ languages),
information about what programs make sense needs to be communicated in other
ways. \textbf{Example}: many libraries have documentation with examples. Great. But if
one of the examples has the line read(message, peer), it’s often difficult to
ascertain whether a related expression, like read(fileChunkStream(file1), peer),
is also valid. Types provide immediate answers to these questions, especially
for someone trained at reading typeful APIs.


\subsection{Disadvantages}

Like any formalism, types require some investment up front to become fluent in.
It is a small, but very important point since Clojure has a dynamic nature and
good for rapid and live development and unnecessary formalism can radically
decrease benefits of using such language.

Type errors are frequently poor. Thus, even though errors are caught sooner, the
way they are reported to the user can be frustratingly opaque, which isn’t good
for programmer motivation. In the world of static typing enthusiasts, there’s
also a segment of people who will state or imply that anyone who gets frustrated
by the user experience of fixing type errors is dumb, not a “real programmer”,
etc. There are also complexity apologists who will defend the “type error user
experience”.

Static typing is a constraint on your program’s structure. How limiting or
liberating these constraints are is up for debate, but some people will argue
it’s a big deal. Some tasks, especially around generic programming, can be very
easily expressed in a dynamic language, but require more machinery in a static
language. For instance, a generic serialization library can be written in a
dynamic language, without anything fancy, but providing the same thing in a
static language requires more machinery (see for instance Haskell’s generics
support), and is sometimes more complicated to use.

It can be difficult to assign static types to some programs, and learning how
best to carve a program up into types is a skill that takes years to master.
This is sort of a subtle point: when programming in a static language, you
always have a choice about what information you encode in the types, and how you
encode things. For instance, you can create and use a nonempty list type and
have the compiler check it statically or you can let nonemptiness be a dynamic
property that you have to check for at runtime. You can have lazy, potentially
infinite strings be a distinct type from strict, finite strings, or you can lump
both concepts into the same type and so on. In a static language, you do always
have the option of building a less typeful API, where less is enforced by the
types, but it is often tempting to spend more time encoding things statically
(and then proving things to the typechecker) than would be saved by avoidance of
potential future bugs. With experience, you develop a good sense for what is
worth tracking statically and what to keep dynamic, but newcomers to static
languages can make bad tradeoffs here, which in turn contributes to needless
complexity in the language’s library ecosystem. Haskell has this problem IMO,
even though on the whole, I find Haskell has some very high quality libraries.

To put this last point another way, with static types, you have to make more
choices, and you may have to make these choices at a point in time where you’re
unsure how to decide, or when the choice feels distracting.

\section{Optional annotations}

Clojure is a dynamic language. Among other things this means that type
annotations are not required for code to run. While Clojure has some support for
type hints, they are not an enforcement mechanism, nor comprehensive, and are
limited to communicating information to the compiler to aid in efficient code
generation. Clojure gets runtime checking of a richer set of types by the JVM
itself.

However it has always been a guiding principle of Clojure, widely valued and
practiced by the community, to simply represent information as data. Thus
important properties of Clojure systems are represented and conveyed by the
shape and other predicative properties of the data, not captured or checked
anywhere since the runtime types are indistinguishable heterogeneous maps and
vectors.

Documentation strings can be used to communicate with human consumers, but they
can’t be leveraged by programs or tests, i.e. they have minimal power. Users
have turned to various libraries such as Schema and Herbert to get more powerful
specifications.

Most systems for specifying structures conflate the specification of the key set
(e.g. of keys in a map, fields in an object) with the specification of the
values designated by those keys. I.e. in such approaches the schema for a map
might say :a-key’s type is x-type and :b-key’s type is y-type. This is a major
source of rigidity and redundancy.

In Clojure we gain power by dynamically composing, merging and building up maps.
We routinely deal with optional and partial data, data produced by unreliable
external sources, dynamic queries etc. These maps represent various sets,
subsets, intersections and unions of the same keys, and in general ought to have
the same semantic for the same key wherever it is used. Defining specifications
of every subset/union/intersection, and then redundantly stating the semantic of
each key is both an antipattern and unworkable in the most dynamic cases.

Many users, especially beginners, are frustrated and challenged by the error
messages produced by hand-written parsing and destructuring code, especially in
macros where there are two contexts of execution (the macro runs at compile time
and its expansion at runtime, either of which could fail due to user error).
This has led to a call for 'macro grammars', but in fact macros are just
functions of data $\rightarrow$ data and any solution for data validation and
destructuring should work as well for them as for any other functions. I.e.
macros are an instance of the problems above.

Finally, in all languages, dynamic or not, tests are essential to quality. Too
many critical properties are not captured by common type systems. But manual
testing has a very low effectiveness/effort ratio. Property-based, generative
testing, as implemented for Clojure in test.check, has proved to be far more
powerful than manually written tests.

A standard approach is needed. In short, Clojure has no standard, expressive,
powerful and integrated system for specification and testing. prismatic.schema,
clojure.spec are solutions for most of the problems mentioned above. They have
some advantages and disadvantages discussed below.

\subsection{prismatic.schema}

One of the difficulties with bringing Clojure into a team is the overhead of
understanding the kind of data (e.g., list of strings, nested map from long to
string to double) that a function expects and returns. While a full-blown type
system is one solution to this problem, there are also lighter weight solution:
prismatic.schema.

Schema is a rich language for describing data shapes, with a variety of features:
\begin{itemize}
\item Data validation, with descriptive error messages of failures (targeted at
  programmers)
\item Annotation of function arguments and return values, with optional runtime
  validation Schema-driven data coercion, which can automatically, succinctly,
  and safely convert complex data types (see the Coercion section below)
\item Schema also supports experimental clojure.test.check data generation from
  Schemas, as well as completion of partial datums, features we've found very
  useful when writing tests
\item Schema is also built into plumbing and fnhouse libraries, which illustrate
  how we build services and APIs easily and safely with Schema
\end{itemize}

Examples below explains most features of prismatic.schema.

\begin{minted}{clojure}
(ns schema-examples
  (:require [schema.core :as s
             :include-macros true ;; cljs only
             ]))

(def Data
  "A schema for a nested data type"
  {:a {:b s/Str
       :c s/Int}
   :d [{:e s/Keyword
        :f [s/Num]}]})

(s/validate
  Data
  {:a {:b "abc"
       :c 123}
   :d [{:e :bc
        :f [12.2 13 100]}
       {:e :bc
        :f [-1]}]})
;; Success!

(s/validate
  Data
  {:a {:b 123
       :c "ABC"}})
;; Exception -- Value does not match schema:
;;  {:a {:b (not (instance? java.lang.String 123)),
;;       :c (not (integer? "ABC"))},
;;   :d missing-required-key}
\end{minted}

\begin{minted}{clojure}
;; s/Any, s/Bool, s/Num, s/Keyword, s/Symbol, s/Int, and s/Str are cross-platform schemas.

(s/validate s/Num 42)
;; 42
(s/validate s/Num "42")
;; RuntimeException: Value does not match schema: (not (instance java.lang.Number "42"))

(s/validate s/Keyword :whoa)
;; :whoa
(s/validate s/Keyword 123)
;; RuntimeException: Value does not match schema: (not (keyword? 123))

;; On the JVM, you can use classes for instance? checks
(s/validate java.lang.String "schema")

;; On JS, you can use prototype functions
(s/validate Element (js/document.getElementById "some-div-id"))
\end{minted}


\subsection{clojure.spec}

% Docs are not enough
% Map specs should be of keysets only
% Manual parsing and error reporting is not good enough
% Generative testing and robustness

Clojure.spec has all features mentioned below, but has some advantages over the
prismatic.schema, because it will became a part of clojure in next release of
the language (1.9), it developed by core team that means that it has more
clojure way decisions and more tightly integrated with other parts of the
language, moreover it probably will be standard of optional annotations for
Clojure in the future. It is a most suitable candidate for implementation
alternatives of static type checking advantages, that is why let's look at what
can be done with clojure.spec and how to use it. Small glossary provided below:


% Yet property based testing requires the definition of properties, which require
% extra effort and expertise to produce, and which, at the function-level, have
% substantial overlap with function specifications. Many interesting properties at
% the function level would already be captured by structural+predicative specs.
% Ideally, specs should integrate with generative testing and provide certain
% categories of generative tests 'for free'.

\begin{itemize}
\item Conform. Conform is the basic operation for consuming specs, and does both
  validation and conforming/destructuring. Note that conforming is 'deep' and
  flows through all of the spec and regex operations, map specs etc. Since nil
  and false are legitimate conformed values, conform returns the distinguished
  \texttt{:clojure.spec/invalid} when a value cannot be made to conform. valid?
  can be used instead as a fully-boolean predicate.

\item Explain. When a value fails to conform to a spec you can call explain or
  explain-data with the same spec+value to find out why. These explanations are
  not produced during conform because they might perform additional work and
  there is no reason to incur that cost for non-failing inputs or when no report
  is desired. An important component of explanations is the path. explain
  extends the path as it navigates through e.g. nested maps or regex patterns,
  so you get better information than just the entire or leaf value. explain-data
  will return a map of paths to problems.

\item Path. Due to the fact that all branching points in specs are labeled, i.e.
  map keys, choices in or and alt, and (possibly elided) elements of cat, every
  subexpression in a spec can be referred to via a path (vector of keys) naming
  the parts. These paths are used in explain, gen overrides and various error
  reporting.

\item Spec. A specification is about how something 'looks', but is, most
  importantly, something that is looked at. Specs should be readable, composed
  of 'words' (predicate functions) programmers are already using, and integrated
  in documentation. Unify specification in its various contexts Specs for data
  structures, attribute values and functions should all be the same and live in
  a globally-namespaced directory. Maximize leverage from specification effort.

\end{itemize}

Writing a spec should enable automatic:

\begin{itemize}
\item Validation
\item Error reporting
\item Destructuring
\item Instrumentation
\item Test-data generation
\item Generative test generation
\item Minimize intrusion
\end{itemize}

Minor modifications to doc and macroexpand will allow independently written
specs to adorn fn/macro behavior without redefinition.

Keep map (keyset) specs separate from attribute (key$\rightarrow$value) specs.
Encourage and support attribute-granularity specs of namespaced keyword to
value-spec. Combining keys into sets (to specify maps) becomes orthogonal, and
checking becomes possible in the fully-dynamic case, i.e. even when no map spec
is present, attributes (key-values) can be checked.

Programmers suffer greatly when they redefine things while keeping the names the
same. Yet some changes are compatible and some are breaking, and most tools
can’t distinguish. Use constructs like set membership and regular expressions
for which compatibility can be determined, and provide tools for compatibility
checking (while leaving general predicate equality out of scope).

There is no reason to limit our specifications to what we can prove, yet that is
primarily what type systems do. There is so much more we want to communicate and
verify about our systems. This goes beyond structural/representational types and
tagging to predicates that e.g. narrow domains or detail relationships between
inputs or between inputs and output. Additionally, the properties we care most
about are often those of the runtime values, not some static notion. Thus spec
is not a type system, but static type checking using clojure.spec partially
implemented by spectrum library.


All programs use names, even when the type systems don’t, and they capture
important semantics. Int x Int y Int z just isn’t good enough (is it
length/width/height or height/width/depth?). So spec will not have unlabeled
sequence components or untagged union bindings. The utility of this becomes
evident when spec needs to talk to users about specs, e.g. in error reporting,
and vice versa, e.g. when users want to override generators in specs. When all
branches are named, you can talk about parts of specs using paths.

Clojure supports namespaced keywords and symbols. Note here we are just talking
about namespace-qualified names, not Clojure namespace objects. These are
tragically underutilized and convey important benefits because they can always
co-reside in dictionaries/dbs/maps/sets without conflict. spec will allow (only)
namespace-qualified keywords and symbols to name specs. People using namespaced
keys for their informational maps (a practice we’d like to see grow) can
register the specs for those attributes directly under those names. This
categorically changes the self-description of maps, particularly in dynamic
contexts, and encourages composition and consistency.

In Lisps (and thus Clojure), code is data. But data is not code until you define
a language around it. Many DSLs in this space drive at a data representation for
schemas, but predicative specs have an open and large vocabulary, and most of
the useful predicates already exist and are well known as functions in the core
and other namespaces, or can be written as simple expressions. Having to
'datafy', possibly renaming, all of these predicates adds little value, and has
a definite cost in understanding precise semantics. spec instead leverages the
fact that the original predicates and expressions are data in the first place
and captures that data for use in communicating with the users in documentation
and error reporting. Yes, this means that more of the surface area of
clojure.spec will be macros, but specs are overwhelmingly written by people and,
when composed, manually so.

As per above, maps defining the details of the values at their keys is a
fundamental complecting of concerns that will not be supported. Map specs detail
required/optional keys (i.e. set membership things) and keyword/attr/value
semantics are independent. Map checking is two-phase, required key presence then
key/value conformance. The latter can be done even when the
(namespace-qualified) keys present at runtime are not in the map spec. This is
vital for composition and dynamicity.


Invariably, people will try to use a specification system to detail
implementation decisions, but they do so to their detriment. The best and most
useful specs (and interfaces) are related to purely information aspects. Only
information specs work over wires and across systems. We will always prioritize,
and where there is a conflict, prefer, the information approach.

There are very few bottom notions in this space and we will endeavor to stick to
them. There are few distinct structural notions - a handful of atomic types,
sequential things, sets and maps. Unsurprisingly, these are the Clojure data
types and fundamental ops will be provided only for these. Similarly there are
mathematical tools for talking about these - set logic for maps and regular
expressions for sequences - that have valuable properties. We will prefer these
over ad hoc solutions.

The generative testing underpinning of spec will leverage test.check and not
reinvent it. But spec users should not need to know anything about test.check
until and unless they want to write their own generators or supplement spec's
generated tests with further property-based tests of their own. There should be
no production runtime dependency on test.check.

The basic idea is that specs are nothing more than a logical composition of
predicates. At the bottom we are talking about the simple boolean predicates you
are used to like int? or symbol?, or expressions you build yourself like \texttt{\#(<
42 \% 66)}. Spec adds logical ops like spec/and and spec/or which combine specs
in a logical way and offer deep reporting, generation and conform support and,
in the case of spec/or, tagged returns.

Specs for map keysets provide for the specification of required and optional key
sets. A spec for a map is produced by calling keys with \texttt{:req} and
\texttt{:opt} keyword arguments mapping to vectors of key names. \texttt{:req}
keys support the logical operators and and or.

\begin{minted}{clojure}
(spec/keys :req [::x ::y (or ::secret (and ::user ::pwd))] :opt [::z])
\end{minted}

One of the most visible differences between spec and other systems is that there
is no place in that map spec for specifying the values e.g. \texttt{::x} can
take. It is the (enforced) opinion of spec that the specification of values
associated with a namespaced keyword, like \texttt{:my.ns/k}, should be
registered under that keyword itself, and applied in any map in which that
keyword appears. There are a number of advantages to this:

\begin{itemize}
\item It ensures consistency for all uses of that keyword in an application where all
  uses should share a semantic

\item It similarly ensures consistency between a library
  and its consumers

\item It reduces redundancy, since otherwise many map specs would need to make
  matching declarations about k Namespaced keyword specs can be checked even
  when no map spec declares those keys

\end{itemize}

This last point is vital when dynamically building up, composing, or generating
maps. Creating a spec for every map subset/union/intersection is unworkable. It
also facilitates fail-fast detection of bad data - when it is introduced vs when
it is consumed.

Of course, many existing map-based interfaces take non-namespaced keys. To
support connecting them to properly namespaced and reusable specs, keys supports
-un variants of \texttt{:req} and \texttt{:opt}.

\begin{minted}{clojure}
(spec/keys :req-un [:my.ns/a :my.ns/b])
\end{minted}

This specs a map that requires the unqualified keys \texttt{:a} and \texttt{:b}
but validates and generates them using specs (when defined) named
\texttt{:my.ns/a} and \texttt{:my.ns/b} respectively. Note that this cannot
convey the same power to unqualified keywords as have namespaced keywords - the
resulting maps are not self-describing.

Specs for sequences/vectors use a set of standard regular expression operators,
with the standard semantics of regular expressions:

\begin{itemize}
\item cat - a concatenation of predicates/patterns
\item alt - a choice of one among a set of predicates/patterns
\item * - zero or more occurrences of a predicate/pattern
\item + - one or more
\item ? - one or none
\item \& - takes a regex op and further constrains it with one or more predicates
\end{itemize}

These nest arbitrarily to form complex expressions. Note that cat and alt
require all of their components be labeled, and the return value of each is a
map with the keys corresponding to the matched components. In this way spec
regexes act as destructuring and parsing tools.

\begin{minted}{clojure}
user=> (require [clojure.spec :as s])
(s/def ::even? (s/and integer? even?))
(s/def ::odd? (s/and integer? odd?))
(s/def ::a integer?)
(s/def ::b integer?)
(s/def ::c integer?)
(def s (s/cat :forty-two #{42}
              :odds (s/+ ::odd?)
              :m (s/keys :req-un [::a ::b ::c])
              :oes (s/* (s/cat :o ::odd? :e ::even?))
              :ex (s/alt :odd ::odd? :even ::even?)))
user=> (s/conform s [42 11 13 15 {:a 1 :b 2 :c 3} 1 2 3 42 43 44 11])
{:forty-two 42,
 :odds [11 13 15],
 :m {:a 1, :b 2, :c 3},
 :oes [{:o 1, :e 2} {:o 3, :e 42} {:o 43, :e 44}],
 :ex {:odd 11}}
\end{minted}

As you can see above, the basic operation for using specs is conform, which
takes a spec and a value and returns the conformed value or
\texttt{:clojure.spec/invalid} if the value did not conform. When the value does
not conform you can call explain or explain-data to find out why it didn’t.

The primary operations for defining specs are s/def, s/and, s/or, s/keys and the
regex ops. There is a spec function that can take a predicate function or
expression, a set, or a regex op, and can also take an optional generator which
would override the generator implied by the predicate(s).

Note however, that def, and, or, keys spec fns and the regex ops can all take
and use predicate functions and sets directly - and do not need them to be
wrapped by spec. spec should only be needed when you want to override a
generator or to specify that a nested regex starts a new, vs being included in
the same pattern.

In order for a spec to be reusable by name, it has to be registered via def. def
takes a namespace-qualified keyword/symbol and a spec/predicate expression. By
convention, specs for data should be registered under keywords and attribute
values should be registered under their attribute name keyword. Once registered,
the name can be used anywhere a spec/predicate is called for in any of the spec
operations.

A function can be fully specified via three specs - one for the args, one for
the return, and one for the operation of the function relating the args to the
return. The args spec for a fn is always going to be a regex that specs the
arguments as if they were a list, i.e. the list one would pass to apply the
function. In this way, a single spec can handle functions with multiple arities.
The return spec is an arbitrary spec of a single value.

The (optional) fn spec is a further specification of the relationship between
the arguments and the return, i.e. the function of the function. It will be
passed (e.g. during testing) a map containing\newline
\mintinline{clojure}{{:args conformed-args :ret conformed-ret}} and will
generally contain predicates that relate those values - e.g. it could ensure
that all keys of an input map are present in the returned map. You can fully
specify all three specs of a function in a single call to fdef, and recall the
specs via fn-specs.

Functions specs defined via fdef will appear when you call doc on the fn name.
You can call describe on specs to get descriptions as forms.

You can use conform directly in your implementations to get its
destructuring/parsing/error-checking. conform can be used e.g. in macro
implementations and at I/O boundaries.

You can selectively instrument functions and namespaces with instrument, which
swaps out the fn var with a wrapped version of the fn that tests the
\texttt{:args} spec. unstrument returns a fn to its original version. You can
generate data for interactive testing with gen/sample.

You can run a suite of spec-generative tests on an entire ns with check. You can
get a test.check compatible generator for a spec by calling gen. There are
built-in associations between many of the clojure.core data predicates and
corresponding generators, and the composite ops of spec know how to build
generators atop those. If you call gen on a spec and it is unable to construct a
generator for some subtree, it will throw an exception that describes where. You
can pass generator-returning fns to spec in order to supply generators for
things spec does not know about, and you can pass an override map to gen in
order to supply alternative generators for one or more subpaths of a spec.

In addition to the destructuring use cases above, you can make calls to conform
or valid? anywhere you want runtime checking, and can make lighter-weight
internal-only specs for tests you intend to run in production.

Note that if you want to nest an independent regex predicate within a regex you
will have to wrap it in a call to spec, else it will be considered a nested
pattern.


\section{Gradual typing}
The popularity of dynamically-typed languages in software development, combined
with a recognition that types often improve programmer productivity, software
reliability, and performance, has led to the recent development of a wide
variety of optional and gradual type systems aimed at checking existing programs
written in existing languages. These include TypeScript and Flow for JavaScript,
Hack for PHP, and mypy for Python among the optional systems, and Typed Racket,
Reticulated Python, and GradualTalk among gradually-typed systems.

One key lesson of these systems, indeed a lesson known to early developers of
optional type systems such as StrongTalk, is that type systems for existing
languages must be designed to work with the features and idioms of the target
language. Often this takes the form of a core language, be it of functions or
classes and objects, together with extensions to handle distinctive language
features.

\subsection{TypedClojure}

Typed Clojure is an optional type system for Clojure. It enables Clojure
programmers to gain greater confidence in the correctness of their code via
static type checking while remaining in the Clojure world, and has acquired
significant adoption in the Clojure community. Typed Clojure repurposes Typed
Racket’s occurrence typing, an approach to statically reasoning about predicate
tests, and also includes several new type system features to handle existing
Clojure idioms. There are three features widely used in Clojure. First,
multimethods provide extensible operations, and their Clojure semantics turns
out to have a surprising synergy with the underlying occurrence typing
framework. Second, Java interoperability is central to Clojure’s mission but
introduces challenges such as ubiquitous null; Typed Clojure handles Java
interoperability while ensuring the absence of nullpointer exceptions in typed
programs. Third, Clojure programmers idiomatically use immutable dictionaries
for data structures; Typed Clojure handles this with multiple forms of
heterogeneous dictionary types. Typed Clojure is now in use by numerous
corporations and developers working with Clojure.

Since Clojure is a language in the Lisp family, there was applied the lessons of
Typed Racket, an existing gradual type system for Racket, to the core of Typed
Clojure, consisting of an extended lambda-calculus over a variety of base types
shared between all Lisp systems. Furthermore, Typed Racket’s occurrence typing
has proved necessary for type checking realistic Clojure programs. However,
Clojure goes beyond Racket in many ways, requiring several new type system
features which we detail in this paper. Most significantly, Clojure supports,
and Clojure developers use, multimethods to structure their code in extensible
fashion. Furthermore, since Clojure is an untyped language, dispatch within
multimethods is determined by application of dynamic predicates to argument
values. Fortunately, the dynamic dispatch used by multimethods has surprising
symmetry with the conditional dispatch handled by occurrence typing. Typed
Clojure is therefore able to effectively handle complex and highly dynamic
dispatch as present in existing Clojure programs. But multimethods are not the
only Clojure feature crucial to type checking existing programs. As a language
built on the Java Virtual Machine, Clojure provides flexible and transparent
access to existing Java libraries, and Clojure/Java interoperation is found in
almost every significant Clojure code base. Typed Clojure therefore builds in an
understanding of the Java type system and handles interoperation appropriately.
Notably, null is a distinct type in Typed Clojure, designed to automatically
rule out null-pointer exceptions. An example of these features is given below:

\begin{minted}{clojure}
(ann pname [(U File String) -> (U nil String)])
(defmulti pname class) ; multimethod dispatching on class of argument
(defmethod pname String [s] (pname (new File s))) ; String case
(defmethod pname File [f] (.getName f)) ; File case, static null check
(pname "STAINS/JELLY") ;=> "JELLY" :- (U nil Str)
\end{minted}

Here, the pname multimethod dispatches on the class of the argument—for Strings,
the first method implementation is called, for Files, the second. The String
method calls a File constructor, returning a non-nil File instance—the getName
method on File requires a non-nil target, returning a nilable type. Other
examples and more detailed theoretical explanation can be found in
\cite{bonnaire2016practical}.

Finally, flexible, high-performance immutable dictionaries are the most common Clojure
data structure. Simply treating them as uniformly-typed key-value mappings would
be insufficient for existing programs and programming styles. Instead, Typed
Clojure provides a flexible heterogenous map type, in which specific entries can
be specified. While these features may seem disparate, they are unified in
important ways. First, they leverage the type system mechanisms inherited from
Typed Racket - multimethods when using dispatch via predicates, Java
interoperation for handling null tests, and heterogenous maps using union types
and reasoning about subcomponents of data. Second, they are crucial features for
handling Clojure code in practice. Typed Clojure’s use in real Clojure
deployments would not be possible without effective handling of these three
Clojure features.

After a 2 year trial, some companies \cite{clojuretypedescape} decided to
disabled type checking. They were supportive of the fundamental ideas presented,
but primarily cited issues with the checker implementation in practice and would
reconsider type checking if they were resolved.

\textbf{Performance.} Rechecking files with transitive dependencies is expensive
since all dependencies must be rechecked. TypedClojure team conjecture caching
type state will significantly improve re-checking performance, though preserving
static soundness in the context of arbitrary code reloading is a largely
unexplored area. Performance problems is a common for sound gradual typing
\cite{takikawa2016sound}.

\textbf{Library annotations.} Annotations for external code are rarely
available, so a large part of the untyped-typed porting process is reverse
engineering libraries.

\textbf{Unsupported idioms.} While the current set of features is vital to
checking Clojure code, there is still much work to do. For example, common
Clojure functions are often too polymorphic for the current implementation or
theory to account for.



\subsection{Spectrum}
Spectrum is another static analysis library. It is much younger than
ClojureTyped and probably not ready for production use, but it based on
clojure.spec, which will be included in next major release of Clojure
programming language (1.9). It means that every project uses clojure.spec can
get optional static type system almost for free and it doesn't have last two
drawbacks of CoreTyped.


%\section{Draft figures}

% We can include encapsulated PostScript\texttrademark\ figures
% (\texttt{.eps}) in the document and refer to it using a label.
% For example, MUN's logo can be seen in Figure~\ref{fig:MUN_Logo_Pantone}.
% \munepsfig{MUN_Logo_Pantone}{This is MUN's logo}

% Figure~\ref{fig:enrollment} shows a chart of MUN's Fall
% enrollment from 2005 -- 2009.\munfootnote{From \emph{Memorial
% University of Newfoundland --- Fact Book 2009}.}
% \munepsfig[scale=0.50]{enrollment}{MUN Fall Enrollment 2005 -- 2009}
% The figure was created using the \textsf{Calc} spreadsheet application of
% the office suite \textsf{OpenOffice.org}.\munfootnote{This office suite
% can be downloaded at no cost from \texttt{http://openoffice.org/}. Unlike
% other commercial office suites, \textsf{OpenOffice.org} may be legally
% shared with colleagues and fellow students.  There are versions for
% Linux, Microsoft Windows, Mac~OS~X and Solaris.  Also, unlike commercial
% offerings, \textsf{OpenOffice.org} does not require activation using
% registration keys.}  This figure was reduced by 50\%.

% For larger figures, we can use landscape mode to rotate the page
% and display the figure using the \verb+\munlepsfig+ command, as shown
% in Figure~\ref{fig:enrollment-landscape}.  The figure will be the
% only thing on the page when typeset in landscape mode.
% (The figure is reduced to 85\% of its original size.)
% \munlepsfig[scale=0.85]{enrollment-landscape}
% 	{MUN Fall Enrollment 2005 -- 2009 (landscape)}

% Alternatively, if we just want to rotate the figure, but not
% the entire page, we can specify an \texttt{angle} attribute
% in the default argument of the \verb+\munepsfig+ command.
% The result is shown in Figure~\ref{fig:enrollment-rotate}.
% If the figure is too large or if there isn't sufficient
% text, then the figure may appear on its own page.
% \munepsfig[scale=0.30,angle=90]{enrollment-rotate}
% 	{MUN Fall Enrollment 2005 -- 2009 (rotated)}

% Note that all three of the enrollment figures are basically
% the same file, but with different names --- on Linux, they are
% symbolic links to the same file.  The filenames have to be different
% because the reference labels need to be unique.

% Figure~\ref{fig:db-deadlock} shows a Petri net created using the
% \texttt{xfig} program (\texttt{http://www.xfig.org/}) which has
% very good support for \LaTeX.  This figure has been
% reduced to 40\% of its original size.
% \munepsfig[scale=0.40]{db-deadlock}{A deadlocked Petri net}

% We can also create figures of text (such as short code snippets)
% using the \verb+\muntxtfig+ command, as show in Figure~\ref{fig:code}.
% \begin{muntxtfig}[1.0]{code}{Hello World}{0.5\textwidth}
% \begin{verbatim}
% #include <stdio.h>

% int main(int argc, char **argv)
% {
%   printf("Hello world!\n");
%   exit(0);
% }
% \end{verbatim}
% \end{muntxtfig}

% %\section{Draft Tables}

% We can also create tables, as seen by Table~\ref{tab:pop}.  Note that,
% as required by SGS guidelines, the caption for a table appears above the
% table whereas figure captions appear below the figures.  Tables and
% figures can ``float'' --- they may not appear on the page on which they
% are mentioned.  \LaTeX{} tries to handle figure and table placement
% intelligently, but if if you have a lot of them without a reasonable
% amount of surrounding textual content, the figures and tables can
% accumulate towards the end of the chapter.  Generally speaking, if
% there is sufficient text explaining the tables and figures or if the
% tables/figures are relatively small, this may not be a problem.  However,
% if you have a lot of tables or figures, it may be a good idea to put
% them in an appendix and refer to them as the need arises.

% \begin{muntab}{c||c|c|c||c|c|c|}{pop}{Fall Semester Enrollment}
% \hline
% 	& \multicolumn{3}{c||}{Undergraduate}
% 	& \multicolumn{3}{c|}{Graduate} \\
% \cline{2-7}
%      & F/T & P/T & Total & F/T & P/T & Total \\
% \cline{2-7}
% 2004 & 13,191 & 2,223 & 15,414 & 1,308 & 879 & 2,187 \\
% 2005 & 13,184 & 2,143 & 15,327 & 1,375 & 920 & 2,295 \\
% 2006 & 12,809 & 2,224 & 15,033 & 1,373 & 899 & 2,272 \\
% 2007 & 12,634 & 2,155 & 14,789 & 1,403 & 899 & 2,302 \\
% 2008 & 12,269 & 2,208 & 14,477 & 1,410 &1,005& 2,415 \\
% 2009 & 12,382 & 2,323 & 14,705 & 1,567 &1,106& 2,673 \\
% \hline
% \end{muntab}

% Table~\ref{tab:degrees} shows a different table in landscape
% mode.\munfootnote{This data was also taken from the \emph{Memorial
% University of Newfoundland --- Fact Book 2009}.} This is useful if your
% table is too wide for the page.  Tables are double-spaced by default.
% To single-space a table, change the \verb+\baselinestretch+ before
% beginning the table environment.  Remember to restore it after the
% environment has ended.

% \renewcommand{\baselinestretch}{1.0}\normalsize
% \begin{munltab}{lrrrrrrrrrrrr}
% 	{degrees}
% 	{Masters Degrees Conferred by Convocation Session --- 1950 to 2009}
% \cline{2-13}
% 				&
% \multicolumn{2}{|c|}{2009}	&
% \multicolumn{2}{c|}{2008}	&
% \multicolumn{2}{c|}{2007}	&
% \multicolumn{2}{c|}{2006}	&
% \multicolumn{2}{c|}{2006}	&
% \multicolumn{1}{c|}{1950--2004}	&
% \multicolumn{1}{c|}{Total}	\\
% \cline{2-13}
% 	  &
% May & Oct &
% May & Oct &
% May & Oct &
% May & Oct &
% May & Oct & &  \\
% Degrees \\
% \hline
% Master of Applied Science		&  14 &   2 &  15 &   8 &  28 &   1 &  21 &   3 &   3 &   1 &    98 &   194 \\
% Master of Applied Social Psychology     &   1 &   5 &   2 &   5 &   1 &   4 &   0 &   4 &   0 &   4 &    28 &    54 \\
% Master of Applied Statistics            &   0 &   0 &   3 &   1 &   0 &   0 &   1 &   0 &   0 &   0 &    19 &    24 \\
% Master of Arts                          &  37 &  49 &  26 &  43 &  14 &  42 &  14 &  56 &  13 &  44 &   994 & 1,332 \\
% Master of Business Administration       &  14 &  16 &  23 &   6 &  33 &  12 &  33 &  11 &  33 &   8 &   818 & 1,007 \\
% Master of Education                     & 107 &  87 & 120 &  55 & 147 &  74 & 108 &  76 & 113 &  75 & 2,603 & 3,565 \\
% Master of Employment Relations          &   8 &   9 &   5 &   7 &   7 &  14 &   4 &   9 &   3 &   5 &    12 &    83 \\
% Master of Engineering                   &  20 &  19 &  20 &  10 &  16 &  10 &  15 &  13 &   4 &  19 &   440 &   586 \\
% Master of Environmental Science         &   3 &   3 &   3 &   1 &   0 &   1 &   7 &   1 &   3 &   1 &    66 &    89 \\
% Master of Marine Studies                &   2 &   0 &   0 &   1 &   0 &   2 &   2 &   2 &   1 &   2 &    26 &    38 \\
% Master of Music                         &   4 &   1 &   5 &   0 &   3 &   0 &   3 &   0 &   3 &   0 &     7 &    26 \\
% Master of Nursing                       &   7 &   8 &  10 &   4 &  17 &   4 &  23 &   7 &   6 &   1 &   116 &   203 \\
% Master of Oil and Gas Studies           &   0 &   0 &   2 &   0 &   0 &   0 &   0 &   2 &   4 &   0 &     0 &     8 \\
% Master of Philosophy                    &   5 &   4 &   2 &   1 &   5 &   2 &   5 &   3 &   2 &   0 &   112 &   141 \\
% Master of Physical Education            &   0 &   2 &   3 &   0 &   5 &   4 &   3 &   0 &   4 &   4 &    84 &   109 \\
% Master of Public Health                 &   0 &   8 &   0 &   0 &   0 &   0 &   0 &   0 &   0 &   0 &     0 &     8 \\
% Master of Science                       &  40 &  32 &  41 &  19 &  29 &  25 &  35 &  29 &  32 &  23 & 1,653 & 1,958 \\
% Master of Science (Kinesiology)         &   1 &   0 &   4 &   2 &   1 &   2 &   2 &   6 &   4 &   3 &     0 &    25 \\
% Master of Science (Medicine)            &  18 &   7 &  11 &   8 &  10 &   5 &   9 &   9 &   8 &   4 &     0 &    89 \\
% Master of Science (Pharmacy)            &   0 &   0 &   1 &   1 &   0 &   0 &   0 &   0 &   1 &   0 &    16 &    19 \\
% Master of Social Work                   &   4 &  11 &   4 &   5 &   4 &   9 &   9 &   5 &   4 &  10 &   257 &   322 \\
% Master of Women's Studies               &   2 &   0 &   2 &   0 &   1 &   1 &   2 &   3 &   2 &   0 &    20 &    33 \\
% \hline
% \textbf{Total Masters}                  & 287 & 263 & 302 & 177 & 321 & 212 & 296 & 239 & 243 & 204 & 7,369 & 9,913 \\
% \end{munltab}
% \renewcommand{\baselinestretch}{\spacing}\normalsize
